\chapter{Resultados}\label{cha:resultados}

En este capítulo presentamos los resultados obtenidos para el ajuste de parámetros del modelo desarrollado, a partir de las observaciones. 
En primer lugar, presentamos los valores ajustados para las condiciones iniciales según los datos observacionales disponibles en la literatura. A continuación, presentamos nuestros resultados para la distribución espectral de energía de ajuste, así como el mapa de temperatura de brillo proyectado en el cielo. Finalmente, presentamos el perfil de temperatura de brillo en función de la altura.
\begin{comment}
En primer lugar, mostramos la distribución espectral de energía correspondiente a la emisión en radio del \jet. 
Posteriormente, a partir de los datos observacionales reportados por Janssen et al. (2021), ajustamos un mapa de calor de la temperatura de brillo proyectada en el cielo, junto con un perfil de temperatura de brillo en función de la altura. 
Finalmente, comparamos estas distribuciones con los resultados del modelo, variando los parámetros libres hasta obtener un ajuste consistente con las observaciones.
\end{comment}

\section{Parámetros y condiciones iniciales.}

Los valores iniciales de los parámetros dinámicos, geométricos y radiativos del modelo fueron determinados combinando restricciones teóricas con observaciones del objeto analizado. Estos parámetros se eligieron de manera que el modelo reproduzca tanto la potencia total observada del \jet como las características del espectro en radio. 

Los parámetros de la fuente son $d=3.8$ Mpc \citep{Harris2010}, $M_\mathrm{BH} = 5.5\cdot10^7~M_\odot$ \citep{Neumayer2010}, $\theta=48\degree$ \citep{Mueller2011}. La emisión predicha por nuestro modelo es radiación sincrotrón, siendo SSC despreciable. Integramos la emisión del \jet hasta $z < z_\mathrm{max}=10^4 R_g$.

\begin{figure}[h]
    \centering
    \includegraphics[width=0.6\linewidth]{graphics/Resultados/RvZ.jpg}
    \caption[Diámetro del radio interior del \jet contra la altura proyectada en el cielo.]{Diámetro del radio interior del \jet contra la altura proyectada en el cielo. Puntos rojos representan observaciones de \cite{2021NatAs...5.1017J}.}
    \label{fig:result-1}
\end{figure}

Para el ajuste de radios, así como el parámetro de apertura que determina la estructura parabólica del \jet, se utilizaron observaciones de \cite{2021NatAs...5.1017J}. Por medio del perfil de colimación obtenido en dicho trabajo, fijamos el parámetro de apertura $w=0.33$ en la Ec. \eqref{eq:mod-3}, tanto para el radio interno como para el externo.

La velocidad del flujo está mediada por el factor de Lorentz en función de la altura, Ec. \eqref{eq:mod-5}

\begin{equation}\label{eq:result-1}
    \Gamma_\mathrm{j}(z) = \Gamma_{\mathrm{j},0} \left(\frac{z}{z_0}\right)^g,
\end{equation}
tomamos el \jet casi no relativista en la base, de modo que $\Gamma_{j,0}\approx1$. Para el valor del factor de potencias tomamos $g=0.13.$

\textcolor{red}{No estoy seguro que mas decir aca.}\

\section{Distribución espectral de energía.}

Para caracterizar la emisión del \jet modelado calculamos el flujo en función de la altura para cada frecuencia, según lo descrito en la Sección \ref{subsec:flujo}. Luego, integrando las Ecuaciones \eqref{eq:mod-42},\eqref{eq:mod-43}, \eqref{eq:mod-44} en la altura del \jet podemos calcular el flujo total en función de la frecuencia.
\begin{figure}[h!]
    \centering
    \includegraphics[width=0.75\linewidth]{graphics/Resultados/dFdz.png}
    \caption{Flujo emitido por celda en funci\'on de la frecuencia.}
    \label{fig:result-2}
\end{figure}
\begin{figure}[h!]
    \centering
    \includegraphics[width=0.75\linewidth]{graphics/Resultados/fluxTot.png}
    \caption{Flujo integrado emitido por el \jet en funci\'on de la frecuencia}
    \label{fig:result-3}
\end{figure}

En las Figuras \ref{fig:result-2} y \ref{fig:result-3} se muestra la distribución espectral de energía (SED) por celda del \jet, así como el flujo total integrado a lo largo de su eje. Ambas figuras fueron obtenidas mediante el modelo de emisión sincrotrón desarrollado en el capítulo anterior. Dicho modelo reproduce adecuadamente la forma general del espectro, caracterizada por una pendiente suave en el rango de radio a óptico y una caída pronunciada hacia frecuencias más altas, en concordancia con un mecanismo de enfriamiento radiativo dominante.

Esta emisión se origina a partir de la distribución de partículas en función de la energía y la altura en el \jet, $N(E,z)$, la cual se calcula a partir de la Ecuación \ref{eq:mod-20}. Para ello, se adopta una función de inyección de partículas según la Ecuación \ref{eq:mod-17}, con un índice espectral $p = 2.3$.

En la Figura \ref{fig:result-2} se observa que el pico de emisión se desplaza hacia frecuencias progresivamente más bajas al aumentar la altura en el \jet. Este comportamiento sugiere una pérdida gradual de energía de las partículas, consistente con un escenario dominado por procesos de enfriamiento sincrotrón y adiabaticos a medida que el plasma se expande y se aleja de la base del \jet.

Podemos ver en la Figura \ref{fig:result-3} que, a frecuencias más altas, el flujo presenta una caída con pendiente de $\sim -1.15$. Este valor de pendiente es consistente con el esperado para una población de partículas con un índice de energía $p \simeq 2.3$, en concordancia con la función de inyección adoptada en el modelo.


\section{Mapa de temperatura de brillo en el cielo.}

Adicionalmente realizamos un mapa de temperatura de brillo en funcion de las coordenadas en el plano del cielo $(x,y)$ para nuestro modelo. En el caso de fuentes resueltas, como Cen A, este tipo de mapas sintéticos constituye una herramienta complementaria para contrastar los resultados del modelo con los mapas de flujo observados y evaluar la validez de los parámetros físicos adoptados.

\begin{figure}[h!]
    \centering
    % --- Row 1 ---
    \begin{subfigure}[b]{0.49\textwidth}
        \centering
        \includegraphics[width=\textwidth]{graphics/Resultados/Mapa_Tb_0-135.jpeg}
        \caption{Mapa de temperatura de brillo en el cielo, $R_\mathrm{in,0}=0$}
        \label{fig:map1}
    \end{subfigure}
    \hfill
    \begin{subfigure}[b]{0.49\textwidth}
        \centering
        \includegraphics[width=\textwidth]{graphics/Resultados/Mapa_Tb_75-135.jpeg}
        \caption{Mapa de temperatura de brillo en el cielo, $R_\mathrm{out,0}=135R_g~R_\mathrm{in,0}=75R_g$}
        \label{fig:map2}
    \end{subfigure}

    \vspace{0.5cm}

    % --- Row 2 ---
    \begin{subfigure}[b]{0.49\textwidth}
        \centering
        \includegraphics[width=\textwidth]{graphics/Resultados/SQRT_T_brightness_(230GHz).png}
        \caption{Mapa de temperatura de brillo en el cielo, $R_\mathrm{out,0}=135R_g~R_\mathrm{in,0}=100R_g$}
        \label{fig:map3}
    \end{subfigure}
    \hfill
    \begin{subfigure}[b]{0.49\textwidth}
        \centering
        \includegraphics[width=\textwidth]{graphics/Resultados/Mapa_Tb_130-135.jpeg}
        \caption{Mapa de temperatura de brillo en el cielo, $R_\mathrm{out,0}=135R_g~R_\mathrm{in,0}=130R_g$}
        \label{fig:map4}
    \end{subfigure}

    \caption{Mapas de temperatura de brillo para diversos valores de $R_\mathrm{out,0}$, $R_\mathrm{in,0}$}
    \label{fig:result-8}
\end{figure}

La Figura~\ref{fig:result-8} presenta los mapas obtenidos para distintos valores de $R_\mathrm{in,0}$ y $R_\mathrm{out,0}$. En particular, la Figura~\ref{fig:map1} muestra el caso de un \jet completamente lleno en su base (sin cavidad interna). La emisión resultante es simétrica en torno al eje, con un máximo de brillo localizado muy cerca de la base y un decaimiento progresivo a lo largo del eje del flujo.

En contraste, las Figuras~\ref{fig:map2}, \ref{fig:map3} y~\ref{fig:map4} corresponden a modelos con distintos cocientes $R_\mathrm{out,0}/R_\mathrm{in,0}$. En la Figura~\ref{fig:map2} comienza a observarse la aparición de una zona central menos brillante, lo que sugiere un \jet parcialmente hueco. A medida que el cociente $R_\mathrm{out,0}/R_\mathrm{in,0}$ disminuye, el brillo máximo tiende a concentrarse en una capa externa, dando lugar a una morfología en la que la emisión se confina a una franja delgada alrededor del eje. En estos casos, la temperatura de brillo máxima disminuye ligeramente, lo que indica una redistribución del flujo radiado. Finalmente, en la Figura~\ref{fig:map4} la emisión central prácticamente desaparece y el perfil longitudinal se extiende más en altura, aunque con una intensidad total menor.


En la Figura \ref{fig:result-5} se presenta un mapa sintético de la emisión del \jet a 230 GHz, en concordancia con las observaciones reportadas por \cite{2021NatAs...5.1017J}. Para obtener dicho mapa, se integró la emisión del \jet proyectada en el plano del cielo y posteriormente se convolucionó con un haz gaussiano (ver Sec. \ref{Sec:Convolucion}), reproduciendo así las condiciones de observación interferométrica. 

\begin{figure}[H]
    \centering
    \includegraphics[width=0.7\linewidth]{graphics/Resultados/SQRT_T_brightness_(230GHz).png}
    \caption[Mapa de temperatura de brillo proyectado en coordenadas $x,~z$ en el plano del cielo]{Mapa de temperatura de brillo proyectado en coordenadas $x,~z$ en el plano del cielo resultante del ajuste a las observaciones de Cen A.}
    \label{fig:result-5}
\end{figure}

Se observa en la figura que el fenómeno de bordes brillantes mencionado en la Sec. \ref{ssec:estructura}. Los bordes del \jet presentan una temperatura de brillo distinta a la de la región interior, con un incremento notable en el límite del borde interno. Es importante considerar el tamaño del haz gaussiano utilizado para el modelado de la observación, ya que este puede modificar fuertemente el mapa resultante (ver Fig. \ref{fig:result-9})

\begin{figure}[H]
    \centering
    \begin{subfigure}[b]{0.49\textwidth}
        \centering
        \includegraphics[width=\textwidth]{graphics/Resultados/Mapa_Tb_convolucionA.jpeg}
    \end{subfigure}

    \vspace{0.5cm}

    % --- Row 2 ---
    \begin{subfigure}[b]{0.49\textwidth}
        \centering
        \includegraphics[width=\textwidth]{graphics/Resultados/Mapa_Tb_convolucionB.jpeg}
    \end{subfigure}
    \hfill
    \begin{subfigure}[b]{0.49\textwidth}
        \centering
        \includegraphics[width=\textwidth]{graphics/Resultados/Mapa_Tb_convolucionC.jpeg}
    \end{subfigure}

    \caption[Mapas de temperatura de brillo para haces gaussianos de convoluci\'on.]{Mapas de temperatura de brillo para haces gaussianos de convoluci\'on. El haz utilizado para la convoluci\'on se encuentra representado por la elipse gris en la esquina superior izquierda de cada mapa.}
    \label{fig:result-9}
\end{figure}

\section{Perfil de temperatura de brillo.}

En la Figura \ref{fig:result-7} podemos ver diversos perfiles de temperatura de brillo máxima en función de la altura del \jet. Dicha figura incluye ajustes para el cociente entre el radio interno y externo (Fig. \ref{fig:sub1}), pruebas en variación de parámetros en z (Fig. \ref{fig:sub2}), ajuste de la potencia total del \jet  (Fig. \ref{fig:sub3}) y pruebas de ajuste en el parámetro $a_\mathrm{inj}$ de la Ecuaci\'on  \eqref{eq:mod-15} (Fig. \ref{fig:sub4}).
\begin{figure}[h!]
    \centering
    % --- Row 1 ---
    \begin{subfigure}[b]{0.49\textwidth}
        \centering
        \includegraphics[width=\textwidth]{graphics/Resultados/T_b_vs_Z_(prueba1).jpeg}
        \caption{Ajuste del perfil de brillo superficial para distintos valores de $R_\mathrm{out,0}/R_\mathrm{in,0}$}
        \label{fig:sub1}
    \end{subfigure}
    \hfill
    \begin{subfigure}[b]{0.49\textwidth}
        \centering
        \includegraphics[width=\textwidth]{graphics/Resultados/T_b_vs_Z_(prueba2).jpeg}
        \caption{Ajuste del perfil de brillo superficial para distintos valores de $z_\mathrm{max}$, $z_\mathrm{max,inj}$ y $z_\mathrm{0}$, en unidades de $R_g$.}
        \label{fig:sub2}
    \end{subfigure}

    \vspace{0.5cm}

    % --- Row 2 ---
    \begin{subfigure}[b]{0.49\textwidth}
        \centering
        \includegraphics[width=\textwidth]{graphics/Resultados/T_b_vs_Z_(prueba3).jpeg}
        \caption{Ajuste del perfil de brillo superficial para distintos valores de $L_\mathrm{j}$}
        \label{fig:sub3}
    \end{subfigure}
    \hfill
    \begin{subfigure}[b]{0.49\textwidth}
        \centering
        \includegraphics[width=\textwidth]{graphics/Resultados/T_b_vs_Z_(prueba4).jpeg}
        \caption{Ajuste del el perfil de brillo superficial para distintos valores de $a_\mathrm{inj}$}
        \label{fig:sub4}
    \end{subfigure}

    \caption{Pruebas de ajuste de parámetros para el perfil de temperatura de brillo.}
    \label{fig:result-7}
\end{figure}

El cociente $R_\mathrm{out,0}/R_\mathrm{in,0}$ determina el grosor de la envoltura del \jet a lo largo de $z$. Dado que adoptamos un mismo parámetro de apertura $w$ para $R_\mathrm{out}$ y $R_\mathrm{in}$ en la Ec.~\eqref{eq:mod-4}, este cociente permanece constante en toda la extensión del \jet. Como se muestra en la Fig.~\ref{fig:sub1}, los modelos con valores pequeños de $R_\mathrm{out,0}/R_\mathrm{in,0}$ producen un \jet más estrecho, con un pico de emisión más compacto e intenso, ya que la radiación se concentra en una región central más densa y caliente. Por el contrario, al aumentar $R_\mathrm{out,0}/R_\mathrm{in,0}$ el \jet se ensancha y la emisión se distribuye sobre un área mayor, lo que disminuye el brillo máximo y suaviza el perfil transversal. El mejor ajuste a las observaciones se obtiene para valores de $R_\mathrm{out,0}/R_\mathrm{in,0}$ cercanos a 1.5, lo que sugiere una estructura del \jet moderadamente ancha en su base.

La Figura~\ref{fig:sub2} muestra el perfil de temperatura de brillo en función de la altura para distintos ajustes de los parámetros que describen la estructura longitudinal del \jet, tales como la altura máxima del \jet, la altura máxima de la región de inyección y la posición correspondiente a la base del \textit{jet}. Se observa que estos parámetros ejercen una influencia significativa sobre la ubicación del máximo en el perfil de temperatura de brillo. En particular, la altura $z$ del máximo coincide aproximadamente con el valor $z_0$ adoptado en el modelo, con una ligera dependencia adicional de $z_\mathrm{max}$. Asimismo, el parámetro $z_\mathrm{max}$ afecta de manera notable la magnitud de la temperatura de brillo: \jets más extendidos tienden a presentar valores de temperatura de brillo más bajos debido a la mayor dispersión de la emisión a lo largo del eje. Por último, $z_\mathrm{max,inj}$ modula la pendiente de ascenso del perfil: para valores más pequeños de este parámetro, la emisión se concentra en regiones cercanas a la base del \jet, dando lugar a un pico más agudo seguido de un descenso más pronunciado.


Finalmente, las Figuras \ref{fig:sub3} y \ref{fig:sub4} muestran el ajuste del perfil para distintos valores de $L_j$ (ver Ec. \eqref{eq:mod-11}) y $a_\mathrm{inj}$, respectivamente. Podemos ver de forma directa que la magnitud del máximo del perfil posee una dependencia directa con el valor de la potencia total del \jet, asi como una ligera dependencia con el valor del parámetro $a_\mathrm{inj}$. A su vez, $a_\mathrm{inj}$ tiene una proporcionalidad inversa con la posicion del máximo de emisión. Esto ocurre debido a que a mayores valores de $a_\mathrm{inj}$ mayor la dependencia inversa de la tasa de inyecci\'on $dL_\mathrm{inj}/dz$ con la altura $z$, por lo cual 

Finalmente, las Figuras~\ref{fig:sub3} y~\ref{fig:sub4} muestran el ajuste del perfil de temperatura de brillo para distintos valores de la potencia total del \jet, $L_\mathrm{j}$ (ver Ec.~\eqref{eq:mod-11}), y del parámetro $a_\mathrm{inj}$, respectivamente. Se observa de manera directa que la magnitud del máximo del perfil posee una fuerte dependencia con el valor de $L_\mathrm{j}$, mientras que muestra una dependencia más débil respecto a $a_\mathrm{inj}$. El parámetro $a_\mathrm{inj}$, a su vez, presenta una relación inversa con la posición del máximo de emisión. Esto posiblemente resulta una consecuencia de que, para valores mayores de $a_\mathrm{inj}$, la tasa de inyección de energía por unidad de altura, $dL_\mathrm{inj}/dz$, decrece más rapidamente con $z$. En consecuencia, la mayor parte de la energía se deposita en regiones más próximas a la base del \jet, concentrando la emisión en zonas internas y desplazando el máximo del perfil hacia alturas menores. Por el contrario, valores más pequeños de $a_\mathrm{inj}$ implican una inyección más extendida a lo largo del \jet, lo que da lugar a un perfil de emisión más amplio y con un pico desplazado hacia alturas mayores.


En la Figura \ref{fig:result-6} se muestra el perfil de temperatura de brillo máxima en función de la altura obtenido para Cen A. Se observa un buen acuerdo con las observaciones hasta aproximadamente $z \sim 200~R_g$, a partir de lo cual la temperatura de brillo observada disminuye rápidamente con la altura. Este comportamiento se condice con lo mostrado en la Figura \ref{fig:marco-2}, que ilustra las observaciones del EHT. En dicha imagen, el \jet parece desviarse a mayores alturas, lo que podría estar asociado a la disminución de la temperatura de brillo en esa región.

\begin{figure}[h!]
    \centering
    \includegraphics[width=0.75\linewidth]{graphics/Resultados/TvZ.jpeg}
    \caption[Perfil de temperatura de brillo $T_\mathrm{b}$ en función de la altura del \jet.]{Perfil de temperatura de brillo $T_\mathrm{b}$ en función de la altura del \jet ajustado a las observaciones de Cen A. En rojo las observaciones del \jet realizadas por \cite{2021NatAs...5.1017J}. En verde las observaciones del contra-\jet.}
    \label{fig:result-6}
\end{figure}


\endinput
