\chapter{Modelo}\label{cha:modelo}

En este capítulo presentamos un modelo semianalítico de emisión para \jets con estructura multizona, desarrollado a partir de \textcolor{red}{cita}. En primer lugar, describimos la estructura general del modelo, incluyendo las escalas adoptadas para caracterizar la inyección de partículas. A continuación, analizamos la evolución de los parámetros geométricos y dinámicos del \textit{jet}, así como su impacto en la potencia total y en la redistribución de energía asociada a la inyección. Posteriormente, se introduce la ecuación de transporte relativista utilizada para describir la evolución de la distribución de electrones relativistas. Finalmente, se detalla la emisión sincrotrón y la expresión empleada para la profundidad óptica en función de los radios interno y externo del modelo.  

\begin{comment}
Para la comparación con datos observacionales se consideran todas las restricciones disponibles en la literatura sobre las condiciones físicas de Centaurus A (Cen~A).  
\end{comment}

\begin{figure}[h!]
    \centering
    \includegraphics[width=0.75\linewidth]{graphics/Modelo/conceptual_program_flow.png}
    \caption{Diagrama del flujo del código para modelo de \jet con estructura}
    \label{fig:mod-1}
\end{figure}

\section{Estructura general del \jet}
En nuestro modelo, postulamos un \jet relativista con estructura diferenciada en funci\'on del radio, distinguiendo la espina (\textit{spine}) para $r < R_\mathrm{in}$ \textcolor{red}{consideramos la espina vacia, explicar mejor esto} y la vaina/envoltura (\textit{sheath}) para $R_\mathrm{in} < r < R_\mathrm{out}$ (ver Fig. \ref{fig:mod-1.5}\textcolor{red}{Hacer un grafico, o encontrar alguno lindo}). El \jet se extiende desde una altura base $z_0$ hasta una distancia máxima $z_{\mathrm{max}} \gg z_0$. En las cercanías de la base, el \jet posee un radio externo inicial $R_0^{\mathrm{out}}$ y un radio interno $R_0^{\mathrm{in}}$. La inyección de partículas relativistas se implementa como un mecanismo continuo y coherente, consistente, por ejemplo, con procesos de reconexión magnética, y se lleva a cabo en la región $z_0 < z < z_{\mathrm{max,inj}}$. Para calcular la evolución de los distintos parámetros físicos, incluyendo la inyección de partículas, dividimos el \jet en segmentos sucesivos. En cada segmento se inyectan electrones, que luego son transportados a lo largo del flujo hacia segmentos más lejanos a medida que se enfrían.

%\begin{figure}[h!]
%    \centering
%    \includegraphics[width=0.75\linewidth]{graphics/Modelo/conceptual_program_flow.png}
%    \caption{Diagrama del flujo del código para modelo de \jet con estructura}
%    \label{fig:mod-1.5}
%\end{figure}

\subsection{Perdidas radiativas.}
Los electrones relativistas pierden energía a través de diferentes procesos, como radiación sincrotrón y pérdidas adiabáticas, asociadas al trabajo realizado por el plasma durante la expansión transversal del \textit{jet}. En el caso de la radiación sincrotrón, la perdida de energía por unidad de tiempo se obtiene directamente a partir de la expresión para la potencia 

\begin{equation}\label{eq:mod-1}
    -\frac{dE}{dt}\Bigg|_{\mathrm{syn}} = \frac{4}{3}\,c\,\sigma_{\mathrm{Th}}\,U_{\mathrm{mag}}\,\beta^2 \gamma^2,
\end{equation}
donde $U_{\mathrm{mag}} = B^2/8\pi$ es la densidad de energía magnética, $\beta = v/c$ es el cociente entre la velocidad de la partícula relativista y la velocidad de la luz, $\gamma$ es el factor de Lorentz del electrón, y $\sigma_{\mathrm{Th}} = \tfrac{8\pi}{3} r_e^2$ es la sección eficaz de Thomson, con $r_e$ el radio clásico del electrón.  

Sustituyendo $U_{\mathrm{mag}}$ y $\sigma_{\mathrm{Th}}$, y considerando $\beta \approx 1$ para electrones relativistas, se obtiene

\begin{equation}\label{eq:mod-2}
    -\frac{dE}{dt}\Bigg|_{\mathrm{syn}} = \frac{4}{9}\,c\,r_e^2\,B^2\,\gamma^2.
\end{equation}

\subsection{Perdidas adiabaticas}

Las pérdidas adiabáticas se refieren a la disminución de la energía interna de un sistema de plasma o gas cuando este se expande sin intercambio de calor con el entorno. Desde un punto de vista hidrodinámico, el término adiabático aparece naturalmente en la ecuación de energía del fluido y actúa incluso en ausencia de un medio externo. En un flujo en expansión, la energía interna se convierte parcialmente en energía cinética del movimiento colectivo del plasma, lo que produce una aceleración del flujo. En contraste, en un flujo de acreción, el proceso inverso conduce a un aumento de la energía interna a expensas del movimiento en masa, generando calentamiento compresional.

El flujo de energía de un \jet consiste predominantemente en la energía interna de los electrones y el movimiento de grupo dominado por iones. A estos componentes se suma una fracción significativa asociada al campo magnético, que puede transportar una parte no despreciable del flujo de energía total en forma de energía Poynting \citep{Blandford1977, Meier2001}. Al moverse en la dirección del \jet, los electrones pierden la parte isotrópica de su energía y ganan velocidad en dicha dirección. Los electrones forman un fluido magnetohidrodinámico junto con protones presentes y campos magnéticos. La energía perdida por procesos adiabáticos es convertida en energía cinética del \textit{jet}, es decir, la velocidad del \jet en estado estacionario aumenta con la distancia al n\'ucleo \citep{Zdziarski2014}. En nuestro modelo, este efecto se considera despreciable, dado que las variaciones en la energía cinética del \jet son pequeñas en comparación con las pérdidas radiativas locales.

La tasa de perdidas adiabaticas esta dada por

\begin{equation}\label{eq:mod-3}
 \frac{dE}{dt}\Bigg|_{\mathrm{Ad}}=-\frac{2}{3}\frac{d\ln(R_\mathrm{out}(z))}{dz}\frac{c\Gamma_\mathrm{j}\beta_\mathrm{j}(z)}{z}\left(E-\frac{(mc^2)^2}{E}\right), 
\end{equation}

Con E la energía en el marco de referencia del fluido.





\section{Geometría y dinámica del \jet}

Presentamos un \jet con geometría cuasi-parabólica multizona, cuya sección transversal está restringida por dos radios definidos como función de la altura $z$:

\begin{equation}\label{eq:mod-4}
    R^{\mathrm{out/in}}(z) = R^{\mathrm{out/in}}_0 \left(\frac{z}{z_0}\right)^w,
\end{equation}

donde $R^{\mathrm{out/in}}_0$ son los radios en la base del \jet y $w$ es el parámetro de apertura que determina la forma cuasi-parabólica del flujo. 
\begin{comment}
    En nuestro modelado de Centaurus A utilizamos $w = 0.33$ para ambos radios, en concordancia con las observaciones del EHT.

\end{comment}

Se espera que, al menos hasta las alturas máximas consideradas en nuestro modelo, el factor de Lorentz del \jet $\Gamma_\mathrm{j}$ siga una ley de potencias respecto de la distancia $z$:

\begin{equation}\label{eq:mod-5}
    \Gamma_\mathrm{j}(z) = \Gamma_{\mathrm{j},0} \left(\frac{z}{z_0}\right)^g,
\end{equation}
con $\Gamma_{\mathrm{j},0} \approx 1$, de modo que el \jet es casi no relativista en su base y se acelera progresivamente a medida que $z$ aumenta. El parámetro $g$ controla la tasa de aceleración: valores mayores implican una aceleración más rápida del flujo relativista cerca de la base del \jet.

\subsection{Ecuaciones de conservación y magnetizaci\'on}
Debido a la ecuación de continuidad relativista, $\nabla_{\alpha}(\rho' u^\alpha)=0$, con $\rho'$ la densidad en el marco comóvil y $u^\alpha$ la tetra velocidad del fluido, el caudal de masa del \textit{jet} $\dot{M}_{\mathrm{j}}$ es constante con $z$. De forma análoga, si consideramos la conservación del tensor de energía–impulso, $\nabla_\mu T^{\mu\nu}=0$, y despreciamos la radiación emitida localmente, la potencia total del \textit{jet} se conserva y puede escribirse como
\begin{equation}\label{eq:mod-6}
    L_{\mathrm{j}}=\int_\Sigma T^{0z}\, d\Sigma_z
    = \Gamma_{\mathrm{j}}\,\dot{M}_{\mathrm{j}}\,c^2\,h'\,(1+\sigma') \equiv \dot{M}_{\mathrm{j}} c^2\,\mu ,
\end{equation}
donde hemos definido el invariante $\mu \;\equiv\; \Gamma_{\mathrm{j}}\,h'\,(1+\sigma')$, $\sigma'$ es la magnetización
\begin{equation}\label{eq:mod-7}
    \sigma'=\frac{B'^2}{4\pi\,\rho' c^2 h'} ,
\end{equation}
y $h'$ la entalpía específica (adimensional),
\begin{equation}\label{eq:mod-8}
    h' \;=\; 1+\frac{u_e'+p_e'}{\rho' c^2} ,
\end{equation}
con $u_e'$ y $p_e'$ la energía interna y la presión (en el marco comóvil) respectivamente.

Igualando $\dot{M}_{\mathrm{j}} = \pi\,R(z)^2\,\Gamma_{\mathrm{j}}(z)\,\beta_{\mathrm{j}}(z)\,\rho' c$, se obtiene que el módulo del campo magnético en el marco del fluido que resulta
\begin{equation}\label{eq:mod-9}
    B'(z)=\frac{2}{\Gamma_{\mathrm{j}}(z)\,R(z)}\,
    \sqrt{\frac{L_{\mathrm{j}}}{c\,\beta_{\mathrm{j}}(z)}\left(\frac{\sigma'}{1+\sigma'}\right)},
\end{equation}
con
\begin{equation}\label{eq:mod-10}
    R(z)=\sqrt{[R^{\mathrm{out}}(z)]^2-[R^{\mathrm{in}}(z)]^2}
\end{equation}
el radio efectivo de la sección anular transversal a $z$.

De aquí se sigue qu\'e $B'(z)\propto [\Gamma_{\mathrm{j}}(z)\,R(z)]^{-1}$. De aqu\'i obtenemos que al expandirse y acelerarse el \textit{jet}, el campo se debilita por dilución y conversión de energía magnética en energía cinética.

De la Ec. \eqref{eq:mod-6} se deduce que la energía por unidad de flujo de masa, $\mu=L_{\mathrm{j}}/(\dot{M}_{\mathrm{j}} c^2)$, es constante. Cerca de la base, donde el plasma es aproximadamente frío ($h'_0 !\approx! 1$) y el flujo aún no ha adquirido velocidades relativistas, se tiene entonces
\[
    \mu \;\approx\; (1+\sigma'_0) \quad \Rightarrow\quad \sigma'_0 \;\approx\; \mu-1 .
\]
Si el \jet alcanza factores de Lorentz apreciables más adelante (y/o transporta gran potencia con un caudal de masa moderado), necesariamente $\mu\gg1$ y, por tanto, $\sigma'_0\gg1$: al inicio la energía está mayormente en el campo. La aceleración consiste precisamente en convertir energía magnética en cinética, haciendo que $\sigma'$ decrezca mientras $\Gamma_{\mathrm{j}}$ aumenta, manteniendo $\mu=\Gamma_{\mathrm{j}} h' (1+\sigma')$ constante.



\section{Potencia del \jet e inyección de partículas}

En esta sección exploramos en detalle los distintos componentes que contribuyen a la potencia total de un \jet relativista, así como los mecanismos responsables de la inyección y aceleración de partículas dentro del flujo. La potencia del \jet constituye un parámetro fundamental para caracterizar la eficiencia del proceso de extracción de energía desde el entorno del agujero negro y su posterior transporte a escalas macroscópicas. Dependiendo de la composición del plasma, la fracción de energía cinética, magnética y radiativa puede variar significativamente, afectando tanto la evolución dinámica del \jet como su emisión observada en distintas bandas del espectro electromagnético.

\subsection{Potencia total del \jet}

Los \jets pueden estar compuestos por un plasma frío de protones y electrones, sumado a una pequeña contribución de partículas relativistas, o por un plasma puramente leptónico de electrones y positrones relativistas \citep{Romero2017}. En general, la potencia de un \jet a una cierta altura desde su base puede separarse en sus distintas componentes: la potencia transportada por "materia fría", que corresponde a protones y electrones asociados no relativistas, $L_\text{p}$; la potencia asociada a electrones relativistas, $L_\text{e}$; y la potencia en el campo magnético, $L_\text{B}$. Luego


\begin{equation}\label{eq:mod-11}
    L_\text{j}=L_\text{p}+L_\text{e}+L_\text{B}.
\end{equation}

Podemos desarrollar las expresiones de estas potencias en función de las variables anteriores. Para la potencia asociada a la masa no relativista $L_p$

\begin{equation}\label{eq:mod-12}
    L_p=\pi R^2\Gamma_\text{j}^2\beta_\text{j}c^3\rho'=\Gamma_\text{j}\dot{M}c^2,
\end{equation}
para la potencia transportada electrones relativistas $L_e$
\begin{equation}\label{eq:mod-13}
    L_e = \pi R^2 \Gamma_j^2 \beta_j c \, (U'_e + p'_e)
\end{equation}
con $U'_e$ la densidad de energía y $p'_e=U'_e/3$ la presi\'on de los electrones relativistas; notamos que podemos escribir la entalpía $h'$ en función de $L_e$ y $L_p$ como $h'=1+L_e/L_p$; finalmente la potencia transportada por el campo magnético $L_B$
\begin{equation}\label{eq:mod-14}
    L_B = \pi R^2 \Gamma_j^2 \beta_j c \, (U'_B + p'_B)
\end{equation}
donde $U'_B = B'^2 / 8\pi$ es la densidad de energía magnética y $p'_B$ la presión magnética; aprovechamos para reescribir la magnetizaci\'on como $\sigma'=L_\text{B}/(L_\text{e}+L_\text{p})$

Dado que en la base la magnetizaci\'on es alta, tenemos que $L_\text{B}$ domina la distribuci\'on de potencia total cerca de la base y disminuye a medida que $z$ aumenta, convirtiendo gradualmente la energ\'ia magnética en aceleración del flujo y energ\'ia de partículas relativistas, y aumentando $L_\text{e}$ y $L_\text{p}$

\begin{figure}[h!]
    \centering
    \begin{subfigure}[b]{0.48\textwidth}
\centering
    \includegraphics[width=\linewidth]{graphics/Modelo/Jet_Power.jpeg}
    \caption[Balance en la potencia a lo largo del \jet]{Figura ilustrativa del balance en las componentes de la potencia a lo largo del \jet}
    \end{subfigure}
    \begin{subfigure}[b]{0.48\textwidth}
\centering
    \includegraphics[width=0.75\linewidth]{graphics/Modelo/sigma_gamma.jpeg}
     \caption[Evolucion de la magnetizaci\'on y el factor de Lorentz a lo largo del \jet]{Figura ilustrativa de la evolucion de la magnetizaci\'on y el factor de Lorentz a lo largo del \jet}
    \end{subfigure}
    \label{fig:modelo-potencia}
\end{figure}

\subsection{Potencia inyectada}
Para caracterizar la inyección en partículas relativistas suponemos que una fracción fija $\varepsilon_\text{e}$ de la potencia total del \jet se transfiere a electrones no térmicos en el rango comprendido entre $z_0$ y $z_\text{max,iny}$. La potencia disponible para inyección resulta entonces  $L_\text{iny} = \varepsilon_\text{e} L_\text{j}$.

Parametrizamos la dependencia espacial de la tasa de inyección mediante
\begin{equation}\label{eq:mod-15}
    \frac{dL_\text{iny}(z)}{dz}=\frac{L_0}{z_0}\left(\frac{z}{z_0}\right)^{-(1+a_\text{iny})}
\end{equation}
donde $a_\text{iny}$ controla la distribución longitudinal de la inyección y $L_0$ es una constante de normalización. La condición de conservación de la potencia inyectada en el intervalo $[z_0, z_\text{max,iny}]$,
\[
    \int_{z_0}^{z_\text{max,iny}} \frac{dL_\text{iny}(z)}{dz}\,dz = L_\text{iny},
\]
permite obtener 
\begin{equation}\label{eq:mod-16}
    L_0=\frac{a_\text{iny}L_\text{iny}}{1-(z_0/z_\text{max,iny})^{a_\text{iny}}},
    \qquad a_\text{iny} \neq 0.
\end{equation}
En el caso particular $a_\text{iny}=0$, la inyección es constante por unidad de $\log(z)$ a lo largo del \jet.  

La potencia total acumulada en electrones no térmicos hasta una distancia $z$ se obtiene integrando la expresión \eqref{eq:mod-16} entre $z_0$ y $z$, lo que conduce a  

\begin{equation}\label{eq:mod-17}
    L_\text{e}(z_0,z)=L_\text{iny}\frac{1-(z_0/z)^{a_\text{iny}}}
    {1-(z_0/z_\text{max,iny})^{a_\text{iny}}}.
\end{equation}

\section{Distribución de partículas relativistas.}

En cada segmento del \jet se inyecta una población de electrones relativistas con una distribución de potencias y un corte exponencial en altas energías. En el sistema comovil,

\begin{equation}\label{eq:mod-18}
    q'(\gamma', z) = q'_0(z)\, \gamma'^{-p} \exp\left[-\frac{\gamma'}{\gamma'_{\max}(z)}\right]
\end{equation}

donde $p$ es el índice espectral (típicamente $2 \lesssim p \lesssim 3$, considerado constante a lo largo del \jet), $\gamma'_{\max}(z)$ es la energía máxima alcanzada por los electrones en la posición $z$, determinada por el balance local entre aceleración y pérdidas, y $q'_0(z)$ es un factor de normalización. Este último se ajusta de manera que la potencia inyectada satisfaga la ecuación de inyección \eqref{eq:mod-16}

\begin{equation}\label{eq:mod-19}
    \frac{dL_{\text{inj}}(z)}{dz} 
= \Gamma_j(z) \pi R^2(z) q'_0(z) 
\int_{\gamma'_{\min}}^{\gamma'_{\max}(z)} 
\gamma'^{-p} e^{-\gamma'/\gamma'_{\max}(z)} d\gamma'.
\end{equation}
Una vez inyectados, los electrones evolucionan en energía y altura a lo largo del \jet. Su distribución estacionaria $n'(\gamma', z)$ obedece la ecuación de transporte relativista,  

\begin{comment}
\begin{equation}\label{eq:mod-20}
    \frac{1}{\pi R^2(z)} 
\frac{\partial}{\partial z} \Big[ \pi R^2(z) \, \Gamma_j(z) \beta_j(z) c \, n'(\gamma', z) \Big]
+ \frac{\partial}{\partial \gamma'} \Big[ \dot{\gamma}'(\gamma',z) \, n'(\gamma',z) \Big]
= q'(\gamma',z),
\end{equation}
\end{comment}
\begin{equation}\label{eq:mod-20}
    \frac{1}{\pi R^2} 
\frac{\partial}{\partial z} \Big[ \pi R^2 \, \Gamma_j \beta_j c \, n'(\gamma') \Big]
+ \frac{\partial}{\partial \gamma'} \Big[ \dot{\gamma}'(\gamma') \, n'(\gamma') \Big]
= q'(\gamma'),
\end{equation}

donde $\dot{\gamma}'$ representa la tasa de variación de energía de las partículas.  

Para simplificar la notación, introducimos $\tilde{N}'(\gamma',z) = \pi R^2(z)\Gamma_j(z)\beta_j(z)c \, n'(\gamma',z)$ como la densidad espectral de partículas transportada por unidad de altura, $\dot{\gamma}'_z = [\Gamma_j(z)\beta_j(z)c]^{-1}\dot{\gamma}'$ como la tasa de pérdidas por unidad de distancia, y $dQ'/dz=\pi R^2(z)q'(\gamma',z)$ como la tasa de inyección de partículas por unidad de altura. Con estas definiciones, la Ec.~\eqref{eq:mod-20} se puede reescribir como

\begin{equation}\label{eq:mod-21}
    \frac{\partial \tilde{N}'(\gamma',z)}{\partial z} 
+ \frac{\partial}{\partial \gamma'} \left[ \dot{\gamma}'_z(\gamma',z) \tilde{N}'(\gamma',z) \right]
= \frac{dQ'(\gamma',z)}{dz}
\end{equation}

La ecuación de transporte no admite en general una solución analítica cerrada. En este trabajo empleamos un esquema numérico de diferencias finitas siguiendo el método de \cite{Park1996}.


\section{Radiación emitida.}

En este trabajo enfocamos el estudio de la emisión en radio presente en \jets relativistas. Dado que en estas longitudes de onda la emisión se encuentra dominada por radiación sincrotrón, nuestro modelo se enfoca en este tipo de proceso.

\subsection{Flujo observado}\label{subsec:flujo}

El flujo observado representa la cantidad de energía que alcanza al observador por unidad de tiempo, área en una dada frecuencia. Matematicamente, el flujo observado se vincula con la intensidad especifica como la integral sobre el \'angulo solido subtendido por la fuente

\begin{equation}\label{eq:mod-22}
F_\nu=\int_\Omega I_\nu \cos(\theta)d\Omega \approx\int_\Omega I_\nu d\Omega.
\end{equation}

Dado que el ángulo proyectado en el cielo es muy pequeño, aproximamos $\cos(\theta)\approx 1$. El angulo solido $\Omega$ puede vincularse con el \'area proyectada en el cielo ocupada por la fuente seg\'un

\begin{equation*}
\Omega=\frac{A}{d^2} \qquad d\Omega=\frac{dA}{d^2},
\end{equation*}
luego

\begin{equation}\label{eq:mod-23}
F_\nu\approx\int_\Omega I_\nu d\Omega.=\int_A I_\nu \frac{dA}{d^2} =\frac{1}{d^2}\int_A I_\nu dxdy,
\end{equation}
donde consideramos $y=z\sin(i)$ la dirección de propagación del \jet proyectada y $x$ perpendicular a esta. En consecuencia, tendremos que

\begin{equation}\label{eq:mod-24}
F_\nu=\frac{\sin(i)}{d^2}\int\int I(x,z)dxdz.
\end{equation}

La intensidad específica $I_\nu$ puede calcularse a partir de la ecuación de transporte radiativo, que describe la variación de la intensidad de la radiación a lo largo de una línea de visión

\begin{equation}\label{eq:mod-25}
    \frac{dI_\nu}{ds}=-\alpha_\nu I_\nu+j_\nu
\end{equation}
donde $s$ es la distancia a lo largo del rayo de propagación, $\alpha_\nu$ es el coeficiente de absorción, que permite calcular la atenuación en la intensidad, y $j_\nu$ es el coeficiente de emisión, que representa la cantidad de intensidad especifica añadida al haz por unidad de longitud. Bajo el supuesto de que tanto $j_\nu$ como $\alpha_\nu$ permanecen aproximadamente constantes a lo largo de la trayectoria, la integración de la Ec.~\eqref{eq:mod-25} conduce a

\begin{equation}\label{eq:mod-26}
    I_\nu(x,z)=\frac{j_\nu}{\alpha_\nu}(1-e^{-\tau_\nu(x,z)})
\end{equation}
con $\tau_\nu(x,z)$ la profundidad óptica.

\subsection{Profundidad óptica}\label{ssec:tau}

\begin{wrapfigure}{l}{0.45\textwidth}
    \includegraphics[width=0.45\textwidth]{graphics/Modelo/cone.png}
    \caption{Diagrama del cruce de un rayo de luz por el espesor del \jet.}
\end{wrapfigure}

En el modelo adoptado, la distribución de partículas se encuentra confinada entre los radios interno y externo definidos en cada altura $z$. Para un rayo que atraviesa un espesor $s(x, z)$ del \jet, la cantidad de materia interceptada depende de la distancia transversal $x$ al eje de simetría.
\begin{wrapfigure}{h}{0.4\textwidth}
    \centering
    \includegraphics[width=0.35\textwidth]{graphics/Modelo/schematic.png}
    \label{fig:schematic}
    \caption{Sección transversal del \jet en el plano del cielo.}
\end{wrapfigure}

Asumimos que, dentro de la región comprendida entre los bordes interno y externo, el coeficiente de absorción $\alpha_\nu$ es constante, y que fuera de esta zona la absorción es despreciable. La longitud efectiva recorrida dentro del medio absorbente depende de la posición transversal $x$, ya que algunos rayos atraviesan solo la envoltura externa, mientras que otros cruzan ambas regiones. En este caso,

\begin{equation}\label{eq:mod-39}
    s(x, z) = \frac{l(x, z)}{\sin i},
\end{equation}

donde $l(x, z)$ es la longitud proyectada del segmento e $i$ es el ángulo entre la dirección de propagación del \jet y la línea de visión del observador:


\begin{equation}
l(x, z) =
\begin{cases}
    2\sqrt{R_{\mathrm{out}}^2(z) - x^2}, 
    & |x| \ge R_\mathrm{in}(z), \\[1ex]
    2\left[\sqrt{R_{\mathrm{out}}^2(z) - x^2}
    - \sqrt{R_{\mathrm{in}}^2(z) - x^2}\right], 
    & |x| < R_\mathrm{in}(z).
\end{cases}
\label{eq:mod-40}
\end{equation}

Con esto podemos resolver la Ec.~\eqref{eq:mod-26} para la {intensidad espectral} $I_\nu$ a lo largo de un trayecto $l(x, z)$, definido por la Ec.~\eqref{eq:mod-40}:

\begin{equation}\label{eq:mod-37}
    I_\nu(x, z) = \frac{j_\nu(z)}{\alpha_\nu(z)}
    \left[ 1 - e^{-\alpha_\nu(z)\, l(x, z)} \right].
\end{equation}
Aqui presentamos $j_\nu$ y $\alpha_\nu$ en el sistema del observador. Podemos transformarlos al sistema comovil utilizando las invariantes relativistas $j_\nu/\nu^2$, $\alpha_\nu~\nu$. Dado que la frecuencia transforma entre sistemas de referencia seg\'un el factor de Doppler (Ec. \eqref{eq:marco-9} y \eqref{eq:marco-10}). Luego el coeficiente de emisi\'on transforma seg\'un

\begin{equation*}
\frac{j'_{\nu'}}{\nu'^2} =\frac{j_\nu}{\nu^2} ,
\end{equation*}
\begin{equation}\label{eq:mod-37b}
\frac{j'_{\nu'}}{\nu'^2} =\frac{j'_{\nu'}}{(\nu/\delta)^2}= \frac{j'_{\nu'} \delta^2}{\nu},
\end{equation}
\begin{equation*}
j'_{\nu'} = j_\nu / \delta^2,
\end{equation*}
mientras que el coeficiente de absorci\'on resulta
\begin{equation*}\label{eq:mod-37c}
\alpha'_{\nu'}~\nu' = \alpha_\nu~\nu,
\end{equation*}
\begin{equation}
\alpha'_{\nu'} = \alpha'_{\nu'}~(\nu/\delta)=\frac{\alpha'_{\nu'}}{\delta}~\nu,
\end{equation}
\begin{equation*}
\alpha'_\nu = \delta\, \alpha_\nu.
\end{equation*}
De estos resultados la expresi\'on de la intensidad en el sistema comovil resulta



\begin{equation}
     I'_\nu'(x, z) = \frac{j'_{\nu'}(z)}{\alpha'_{\nu'}(z)}\, \delta^3
    \left[ 1 - e^{-\frac{\alpha'_{\nu'}(z)}{\delta}\, l(x, z)} \right].
    \label{eq:mod-38}
\end{equation}


\subsection{Coeficientes de emisividad y absorci\'on}
La {emisividad sincrotrón} $j_\nu$ describe la potencia radiada por unidad de volumen, frecuencia y ángulo sólido:

\begin{equation}\label{eq:mod-32}
    j_\nu = \frac{dE}{dt\, d\nu\, dV\, d\Omega}.
\end{equation}

En el sistema comovil, para un valor fijo de $z$ y energía de fotón $E_\gamma=h~\nu'$, la potencia emitida por un único electrón de energía $E'$ en un campo magnético $B'(z)$ puede calcularse como se detalla por la Ec.~\eqref{eq:mod-31} detallada en el Anexo \ref{cha:apx1}. La emisividad total se obtiene multiplicando dicha potencia por la distribución de electrones $N'(E', z)$ e integrando sobre todas las energías disponibles. La importancia de utilizar el sistema comovil reside en que nos permite asumir razonablemente que la emisividad resulta isotropica en este, por lo cual resulta

\begin{equation}
    j'_{\nu'}(z) = \frac{1}{4\pi} 
    \int P(\nu', E', B'(z))\, N'(E', z)\, dE'.
    \label{eq:mod-33}
\end{equation}

La {absorción sincrotrón} se caracteriza mediante el coeficiente $\alpha_\nu$, que cuantifica la atenuación de la radiación al propagarse a través del plasma. Su cálculo en el sistema comovil se basa en la derivada espectral de la distribución de electrones en espacio logarítmico,

\begin{equation}\label{eq:mod-34}
    \frac{d \ln N'}{d \ln E'},
\end{equation}
a partir de la cual, en combinaci\'on con la potencia sincrotron (Ec. \eqref{eq:mod-31}) se construye el integrando

\begin{equation}\label{eq:mod-35}
    P_{\nu'}(\nu', E', B'(z))\, N'(E', z)\, 
    \left( \frac{d \ln N'}{d \ln E'} - 2 \right).
\end{equation}
Asumiendo isotropia (nuevamente, razonable en el sistema comovil) la integración sobre energía, multiplicada por el factor físico $-h^3 c^2 / (8\pi E_\gamma^2)$, conduce a la expresión del coeficiente de absorción en el sistema comóvil:

\begin{equation}
    \alpha'_{\nu'}(z) = 
    -\frac{h^3 c^2}{8 \pi E_\gamma^2}
    \int P_{\nu'}(\nu', E', B'(z))\, N'(E', z)\, 
    \left( \frac{d \ln N'}{d \ln E'} - 2 \right) dE'.
    \label{eq:mod-36}
\end{equation}

\subsection{Regimenes opticos.}

Para calcular el flujo total, integramos la intensidad sobre la coordenada $x$ (Ec.~\ref{eq:mod-24}). En cada segmento del \jet, el tratamiento depende del régimen óptico: delgado, grueso o intermedio (tratamiento general).

\subsubsection*{Tratamiento general}

En el caso general, fuera de los limites de los regímenes ópticamente delgado y grueso, integramos numéricamente para obtener el valor del flujo

\begin{equation*}\label{eq:mod-44}
    \frac{dF_{\nu'}}{dz}
    = \frac{\sin i}{d^2}\,
      \frac{j'_{\nu'}(z)}{\alpha'_{\nu'}(z)}\, \delta^3
      \int_{-R_\mathrm{out}}^{R_\mathrm{out}} (1-e^{-\tau'_{\nu'}(x, z)})\, dx,
\end{equation*}
\begin{equation}
    = \frac{\sin i}{d^2}\,
      \frac{j'_{\nu'}(z)}{\alpha'_{\nu'}(z)}\, \delta^3
      \int_{-R_\mathrm{out}}^{R_\mathrm{out}} \left[ 1 - e^{-\frac{\alpha'_{\nu'}(z)}{\delta \sin(i)}\, l(x, z)} \right] dx,
\end{equation}


\subsubsection*{Caso ópticamente delgado}

La profundidad óptica máxima corresponde al segmento de mayor longitud dentro de la región absorbente, es decir, para $x = R_\mathrm{in}$. En tal caso:

\begin{equation}\label{eq:mod-41}
    \tau_{\nu,\mathrm{max}}(z)
    = \frac{\alpha_\nu(z)}{\sin i}
      \, 2\sqrt{R_{\mathrm{out}}^2(z) - R_{\mathrm{in}}^2(z)}.
\end{equation}

Consideramos que el medio es ópticamente delgado si $\tau_{\nu,\mathrm{max}} < 0.1$. En este régimen, $(1 - e^{-\tau_\nu}) \approx \tau_\nu(x, z)$, y sustituyendo en la Ec.~\ref{eq:mod-24} obtenemos:

\[\]
\begin{equation*}\label{eq:mod-42}
    \frac{dF_{\nu'}}{dz}
    = \frac{\sin i}{d^2}\,
      \frac{j'_{\nu'}(z)}{\alpha'_{\nu'}(z)}\, \delta^3
      \int_{-R_\mathrm{out}}^{R_\mathrm{out}} \tau'_{\nu'}(x, z)\, dx,
\end{equation*}

\begin{equation}
    = \frac{\sin i}{d^2}\, j'_{\nu'}(z)\, \delta^2
      \int_{-R_\mathrm{out}}^{R_\mathrm{out}} l(x, z)\, dx,
\end{equation}

\begin{equation*}
    = \frac{\pi}{d^2}\, j'_{\nu'}(z)\, \delta^2
      \left[R_\mathrm{out}^2(z) - R_\mathrm{in}^2(z)\right].
\end{equation*}

\subsubsection*{Caso ópticamente grueso}

Para un valor medio $x = (R_\mathrm{in} + R_\mathrm{out})/2$, definimos una profundidad óptica promedio $\tau_{\nu,\mathrm{av}}(z) = \alpha'_{\nu'}\, l_\mathrm{av}/(\delta \sin i)$. Si $\tau'_{{\nu'},\mathrm{av}}(z) > 10$, el medio se considera ópticamente grueso y $(1 - e^{-\tau_\nu}) \approx 1$. En este régimen, la contribución al flujo es:

\begin{equation}\label{eq:mod-43}
    \frac{dF_\nu}{dz}
    = \frac{\sin i}{d^2}\,
      \frac{j'_{\nu'}(z)}{\alpha'_{\nu'}(z)}\, \delta^3
      \int_{-R_\mathrm{out}}^{R_\mathrm{out}} dx,
\end{equation}

\begin{equation*}
    = 2R_\mathrm{out}\,
      \frac{\sin i}{d^2}\,
      \frac{j'_{\nu'}(z)}{\alpha'_{\nu'}(z)}\, \delta^3.
\end{equation*}

\section{Convoluci\'on}\label{Sec:Convolucion}

En radioastronomía, un mapa $F(x,y)$ representa una distribución de brillo superficial, en este caso expresada en unidades de flujo por unidad de ángulo sólido. Sin embargo, la resolución finita de los telescopios introduce limitaciones que impiden observar fuentes puntuales de manera perfecta.

La función de dispersión de punto (PSF, por sus siglas en inglés) describe la respuesta de un sistema de imagen ante una fuente puntual u objeto puntual. Esta función se modela comúnmente mediante un haz gaussiano de la forma

\begin{equation}\label{eq:mod-45}
B(x,y) = \exp{\left[-\frac{x^2}{2\sigma_x^2} - \frac{y^2}{2\sigma_y^2}\right]},
\end{equation}
donde $\sigma_x$ y $\sigma_y$ caracterizan la extensión del haz en las direcciones $x$ e $y$, respectivamente.
El mapa observado se obtiene entonces como la convolución del mapa modelado con el haz instrumental.

\subsection{Convolucion discreta.}

Para reproducir las condiciones de observación interferométrica, tomamos el mapa de flujo obtenido (ver Subsec. \ref{subsec:flujo}) y realizamos una convolución discreta sobre las celdas espaciales, dada por

\begin{equation}\label{eq:mod-46}
F_\mathrm{conv}(x',y') \approx \sum_{x,y} F(x,y), B(x-x',y-y'), \Delta x, \Delta y,
\end{equation}
donde $\Delta x$ y $\Delta y$ representan el tamaño de las celdas. Estos valores se definen como $\Delta x, \Delta y = f_\mathrm{Ny}, \sigma_{x,y}$, siendo $f_\mathrm{Ny}$ la frecuencia de muestreo de Nyquist. De esta forma, el flujo observado convolucionado resulta

\begin{equation}\label{eq:mod-47}
F_\mathrm{conv}(x',y') \approx \sum_{x,y} F(x,y),
\exp!\left[-\frac{(x-x')^2}{2\sigma_x^2} - \frac{(y-y')^2}{2\sigma_y^2}\right]
, \Delta x, \Delta y.
\end{equation}

Finalmente, es posible calcular el cociente entre el flujo total modelado y el flujo total observado. Dado el flujo modelado por celda $F(x,y)$, el flujo total se obtiene como

\begin{equation}\label{eq:mod-48}
S = \sum_{x,y} F(x,y), \Delta x, \Delta y.
\end{equation}
Por su parte, el flujo total tras la convolución está dado por

\begin{equation}\label{eq:mod-49}
S_\mathrm{conv} = \sum_{x,y} F(x,y), \frac{\Delta x, \Delta y}{A_\mathrm{haz}},
\end{equation}
donde $A_\mathrm{haz} = 2\pi, \sigma_x, \sigma_y$ corresponde al área del haz gaussiano. El término ${\Delta x, \Delta y}/{A_\mathrm{haz}}$ representa el cociente entre el área de una celda y el área efectiva del haz de observación.

\subsection{Prueba de convolucion sobre dos fuentes puntuales.}

Consideramos dos fuentes puntuales que emiten un flujo de $1~\mathrm{mJy}$ cada una. Se realizó una convolución con un haz gaussiano, suponiendo una anchura a media altura (FWHM, por sus siglas en inglés) de $0.1~\mathrm{mas}$ en el plano del cielo.

\begin{figure}[h]
\centering
\begin{subfigure}[b]{0.48\textwidth}
\centering
\includegraphics[width=\linewidth]{graphics/Modelo/Convolucion_a_dos_puntos.png}
\caption[Mapa de prueba de flujo para dos fuentes puntuales]{Mapa de prueba de flujo para dos fuentes puntuales proyectado en coordenadas $x$ y $z$ en el plano del cielo.}
\label{fig:conv-dos-puntos-S}
\end{subfigure}
\hfill
\begin{subfigure}[b]{0.48\textwidth}
\centering
\includegraphics[width=\linewidth]{graphics/Modelo/Convolucion_a_dos_puntos_Tb.png}
\caption[Mapa de prueba de temperatura de brillo para dos fuentes puntuales]{Mapa de prueba de temperatura de brillo para dos fuentes puntuales proyectado en coordenadas $x$ y $z$ en el plano del cielo.}
\label{fig:conv-dos-puntos-Tb}
\end{subfigure}
\caption[Prueba de convolución sobre dos fuentes puntuales]{Resultados de la convolución sobre dos fuentes puntuales de igual flujo con un haz gaussiano de FWHM $=0.1~\mathrm{mas}$.}
\label{fig:conv-dos-puntos}
\end{figure}

El flujo total previo a la convolución corresponde naturalmente a la suma de los flujos de ambas fuentes, es decir $S = S_1 + S_2 = 2~\mathrm{mJy}$. Tras la convolución, el flujo total obtenido es $S_\mathrm{conv} \approx 1.941~\mathrm{mJy}$, lo que implica un cociente $S_\mathrm{conv}/S \approx 0.9705$. Dicha ligera disminución resulta consistente con el efecto de suavizado introducido por el haz gaussiano, que redistribuye el flujo sobre un área más extensa del plano del cielo, reproduciendo así las limitaciones impuestas por la resolución del instrumento.

