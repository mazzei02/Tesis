\begin{Large}\textbf{Acr\'onimos}\end{Large}

\bigskip

Lista de acr\'onimos utilizados en esta tesis (notar que las siglas usualmente corresponden a las utilizadas en el idioma ingl\'es):

\begin{itemize}

 \item AGN: Núcleos galácticos activos (\textit{Active Galactic Nuclei})
 \item SMBH: Agujeros negros supermasivos (\textit{Supermassive Black Holes})
 \item LINER: Núcleos de emisión de baja ionización (\textit{Low-Ionization Nuclear Emission-Line Region})
 \item BLR: Región de líneas anchas (\textit{Broad Line Region})
 \item NLR: Región de líneas estrechas (\textit{Narrow Line Region})
 \item FR~I: Fanaroff–Riley tipo I
 \item FR~II: Fanaroff–Riley tipo II
 \item Cen~A: Centaurus A (NGC 5128)
 \item BZ: Mecanismo de Blandford–Znajek
 \item BP: Mecanismo de Blandford–Payne
 \item IC: Compton inverso (\textit{Inverse Compton})
 \item SSC: \textit{Synchrotron Self-Compton}
 \item DSA: Aceleración difusiva por choques (\textit{Diffusive Shock Acceleration})
 \item SSA: Auto-absorción sincrotrón (\textit{Synchrotron Self-Absorption})
 \item EHT: \textit{Event Horizon Telescope}
 \item VLBI: Interferometría de muy larga línea de base (\textit{Very Long Baseline Interferometry})
 \item PSF: Función de dispersión de punto (\textit{Point Spread Function})
 \item FWHM: Anchura a media altura (\textit{Full Width at Half Maximum})
 \item SED: Distribución espectral de energía (\textit{Spectral Energy Distribution})
\end{itemize}

 
 
 
 
 
 
 


\end{itemize}
