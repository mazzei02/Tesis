\begin{Large}\textbf{Acr\'onimos}\end{Large}

\bigskip

Lista de acr\'onimos utilizados en esta tesis (notar que las siglas usualmente corresponden a las utilizadas en el idioma ingl\'es):

\begin{itemize}

 
 \item AGN:Núcleos galácticos activos (\textit{Active Galactic Nuclei})
 \item SMBH: Agujeros negros supermasivos (\textit{Supermassive Black Holes})
 \item LINER: Regiones nucleares emisoras de baja ionización (\textit{Low-ionization nuclear emission-line region})
 \item BLR: Región de líneas anchas (\textit{Broad Line Region})
 \item NLR: Región de líneas estrechas (\textit{Narrow Line Region})
 \item FR~I: Fanaroff–Riley tipo I
 \item FR~II: Fanaroff–Riley tipo II
 \item BZ: Blandford-Znajek
 \item BP: Blandford-Payne
 \item IC: Compton inverso (\textit{Inverse Compton})
 \item SSC: \textit{Synchrotron Self Compton}
 \item DSA: Aceleraci\'on difusiva por choques (\textit{Diffusive Shock Acceleration})
 \item SSA: Auto-absorci\'on sincrotrón (\textit{Synchrotron Self Absorption})
 \item EHT: Telescopio \textit{Event Horizont} (\texit{Event Horizont Telescope})
 \item VLBI: Interferometria de muy larga base (\textit{Very Long Baseline Interferometry})
 \item PSF: Función de dispersión de punto (\textit{Point Spread Function}
 \item FWHM: Anchura a media altura (\textit{Full Width at Half Medium}) 
 \item SED: Distribución espectral de energía (\textit{Spectral Energy Distribution})
 
 
 
 
 
 
 


\end{itemize}
