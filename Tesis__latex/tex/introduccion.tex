\chapter{Introducción}\label{cha:introduccion}

La presente Tesis se enmarca en trabajo realizado por el grupo... 
El propósito de este estudio es obtener una caracterización más detallada de...


%=====================================================
%
\section{Contexto}
%
%=====================================================

Contexto del trabajo... 


%=====================================================
%
\section{Objetivos}
%
%=====================================================
Este trabajo de tesis tiene por objetivo principal caracterizar... 

Adicionalmente, dentro de este trabajo se contempla:
\begin{itemize}
    \itemsep0em
    \item Caracterizar X cosa.
    \item Estimar Y parámetro.
    \item Estudiar el posible efecto de Z en coso.
\end{itemize}

%=====================================================
%
\section{Metodología}
%
%=====================================================

En este trabajo se llevó a cabo un análisis detallado de un conjunto de más de un centenar de observaciones realizadas con...
Exploramos los efectos físicos... 

La metodología de trabajo consistió en estudiar ... 

De esta manera también pondremos a prueba hipótesis tales como la existencia de...

%Para ello, utilizamos códigos públicos desarrollados por miembros de la comunidad internacional, tales como \texttt{PRESTO} \citep{Ransom2011} y otros de elaboración propia (\url{https://github.com/lalala/sarasa}).

\bigskip
\bigskip

\noindent La tesis está organizada de la siguiente manera:
%
\begin{itemize}
    \item En el Cap.~\ref{cha:radio} se introducen las principales definiciones y conceptos elementales de...
    Detallamos las características técnicas del instrumental utilizado...
    \item En el Cap.~\ref{cha:ondas} presentamos los resultados obtenidos mediante un análisis de... 
    \item Finalmente, en el Cap.~\ref{cha:conclusiones} presentamos las conclusiones y las perspectivas a futuro de esta línea de investigación.
\end{itemize}
