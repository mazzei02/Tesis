% ----------------------------------
% Preambulo
% ----------------------------------
\documentclass[11pt,a4paper]{book}                     % Tipo de documento y otras especificaciones
\usepackage[hmargin={2.5cm,3cm},vmargin=3cm]{geometry} % Tamaño del área de escritura de la página
\usepackage[provide=*,spanish]{babel}        % Para que los títulos de figuras y tablas estén en español
\addto\captionsspanish{
  \renewcommand{\tablename}{Tabla}
  \renewcommand{\listtablename}{Índice de tablas}
}
%\usepackage[latin1]{inputenc}                          % Para escribir tildes y eñes
\usepackage[utf8]{inputenc}

\usepackage[square,numbers]{natbib}
\usepackage{wrapfig}

\usepackage{natbib}
\bibliographystyle{abbrvnat}
\setcitestyle{authoryear,open={(},close={)}} %Citation-related commands


\usepackage{textcomp}
\usepackage{lmodern}
\usepackage{amsmath}
\usepackage{amssymb}
\usepackage{gensymb}
\usepackage{float}
\usepackage{comment}
\usepackage{graphicx, import}           % Para insertar gráficas
\usepackage{url}
\usepackage[font=small,%
            labelfont=bf,%
            labelsep=period,%
            width=0.915\textwidth]{caption} % Mejorar estilos de los captions
\usepackage{subcaption} % para subfiguras
\usepackage{fancyhdr} % Generar cabeceras con estilo propio

\usepackage{titlesec}                                 % Espacio antes de los section
\titlespacing*{\section}
{0pt}{3.5ex plus 1ex minus .2ex}{2.5ex plus .2ex}
\usepackage[shortlabels]{enumitem}

%\usepackage{enumerate}
\usepackage{siunitx}
\usepackage{array,booktabs}                            % Para insertar tablas
\usepackage{threeparttable}                                  % Para insertar notas al pie en las tablas
\usepackage{rotating}
%\usepackage[]{natbib}
%\usepackage{cite}
%\usepackage[backend=bibtex]{biblatex}
%\usepackage[date=year]{biblatex}

%colorcitos
\usepackage[hidelinks]{hyperref}
\usepackage{xcolor}

\usepackage{afterpage}
\usepackage{graphbox}
\usepackage{float}
\usepackage{ulem}    % Para poder tachar texto horizontalmente

%\hypersetup{
%    colorlinks,
%    linkcolor={green!50!black},
%    citecolor={red!70!black},
%    urlcolor={blue!50!black}
%}
\hypersetup{ colorlinks = true, linkbordercolor = {white}, citecolor={green!50!black}, linkcolor = {red!50!black} }

\newcolumntype{C}{>{$\displaystyle}c<{$}}

\sisetup{output-decimal-marker = {.}}

% Definir el estilo de las cabeceras
\pagestyle{fancy}                                               % Contenido de los encabezados y pies de pagina
\renewcommand{\chaptermark}[1]{\markboth{\thechapter.\ #1}{}}
\renewcommand{\sectionmark}[1]{\markright{\thesection.\ #1}}
\fancyhf{}
\fancyhead[LE]{\small\bfseries\leftmark}
\fancyhead[RO]{\small\bfseries\rightmark}
\fancyfoot[LE,RO]{\bfseries\thepage}
\fancypagestyle{plain}{
  \fancyhf{}
  \renewcommand{\headrulewidth}{0pt}
  \fancyfoot[LE,RO]{\bfseries\thepage}
}

%\setlength{\headheight}{52pt}% ...at least 51.60004pt

\setlength{\bibsep}{0pt plus 0.3ex}

% Comando para generar paginas vacías para two-side
\newcommand{\clearemptydoublepage}%
{\newpage{\pagestyle{empty}\cleardoublepage}}

\interfootnotelinepenalty=10000
\makeatletter
\@addtoreset{footnote}{section}
\makeatother
\renewcommand{\thefootnote}{(\roman{footnote})}

\newcommand{\farcs}{^{\prime\prime}}
\newcommand{\diff}{\;\mathrm{d}}
\newcommand{\seg}{~s$^{-1}$}
\newcommand{\super}{~cm$^{-2}$}
\newcommand{\kms}{~km~s$^{-1}$}
\newcommand{\degr}{^{\circ}}
\newcommand{\sun}{\odot}
\newcommand{\mbeam}{mJy~beam$^{-1}$}
\newcommand{\Ec}[1]{Ec.~(#1)}
\newcommand\numberthis{\addtocounter{equation}{1}\tag{\theequation}}
\newcommand{\jet}{\textit{jet }}
\newcommand{\jets}{\textit{jets }}

\DeclareMathOperator\erf{erf}

\binoppenalty=\maxdimen
\relpenalty=\maxdimen

\setcounter{secnumdepth}{3}
\setcounter{tocdepth}{3}

\def\aap{A\&A}
\def\aaps{A\&AS}
\def\aj{AJ}
\def\apj{ApJ}
\def\apjs{ApJS}
\def\apss{Ap\&SS}
\def\apjl{ApJ}
\def\aplett{Astrophys. Lett.}
\def\araa{ARA\&A}
\def\gca{Geochim. Cosmochim. Acta}
\def\mnras{MNRAS}
\def\nat{Nature}
\def\pasp{PASP}
\def\pasj{PASJ}
\def\bain{Bull. Astron. Inst. Netherlands}
\def\aapr{A\&ARev}
\def\memras{Mem.R.Astron.Soc.}
\def\pasa{PASA}
\def\jrasc{JRASC}
\def\iaucirc{IAU Circ.}
\def\na{New Astron.}
\def\rmxaa{Rev.Mex.A.A.}
\def\rmp{Rev.Mod.Phys.}
\def\ssrv{Space Sci. Rev.}

% ----------------------------------
% Documento
% ----------------------------------
\begin{document}
\begin{comment}
\frontmatter
\thispagestyle{empty}
\begin{center}
    \begin{minipage}{\textwidth}
        \centering
        \includegraphics[totalheight=8cm]{graphics/portada/logo-unlp}
    \end{minipage}
    \\[\intextsep]
    \Large Universidad Nacional de La Plata \\
    Facultad de Ciencias Astron\'omicas y Geofísicas

    \vspace*{\stretch{1}}

    \LARGE Tesis para obtener el grado acad\'emico de\\ Licenciado en Astronom\'ia \\[0.2cm]
    \textsc{Modelo de \jet relativista con estructura aplicado a observaciones de muy alta resolución en radiogalaxias} \\[0.5cm]
    \Large Tom\'as Tob\'ias Mazzei \\

    \vspace*{\stretch{2}}

    \Large Director: Dr. Eduardo Mario Gutierrez \\
    \Large Co-Director: Dra. Florencia Laura Vieyro \\
    
    \vspace*{\stretch{2}}    

    \normalsize \textsc{La Plata, Argentina\\ - Noviembre 2025 -}
\end{center}
\endinput
      % Portada 
\clearemptydoublepage
\chapter{Prefacio}\label{cha:prefacio}

Esta Tesis es presentada como parte de los requisitos para obtener el grado acad\'emico de (Doctor/Lic.) en Astronom\'ia de la Universidad Nacional de La Plata. 
La misma contiene los resultados de las investigaciones
desarrolladas bajo la direcci\'on de los Drs. AA y BB, junto con investigaciones desarrolladas en colaboraci\'on con 
investigadores del grupo CCC, del grupo DD, y otros colegas
internacionales entre los a\~nos 20XX y 20YY.

\vspace*{5cm}

\begin{flushright}
Nombre y Apellido.\\
e-mail: \url{mail@fcaglp.unlp.edu.ar}\\
%Sitio Web: \url{http://newton.fcaglp.unlp.edu.ar}\\
La Plata, (MES) de 202X. \\
\end{flushright}

\endinput
     % Prefacio
\clearemptydoublepage
\chapter{Resumen}\label{cha:resumen}

Este es el resumen que ocupa menos de una carilla.


\newpage

\chapter{Abstract}\label{cha:abstract}

This is the \sout{pencil of Esther Piscore} abstract. 

\endinput
      % Resumen
\clearemptydoublepage
\thispagestyle{empty}
\begin{flushright}
A mi mamá, por la confianza.\\ A Manu, por el apoyo.\\ A mi tío Raúl, por hacer esto posible.
\end{flushright}



\endinput
  % Dedicatoria
\clearemptydoublepage

\chapter{Agradecimientos}

Me resulta difícil imaginar cómo este pequeño resumen de todas las cosas que tengo para agradecer podría hacer justicia a todas las personas que se lo merecen. Por favor, quien lea esto sepa que simplemente el hecho de tomarse el tiempo para leer esta parte de mi trabajo ya demuestra que merece todos mis agradecimientos, y quien no aparezca nombrado a continuación, que sepa disculparme.

Un millón de gracias a Edu, por animarse a ser mi director. Que sepa que hizo un trabajo increíble; no podría haber pedido a alguien mejor. A Flor, a quien no puedo explicarle lo mucho que me ayudó a entender infinitas cosas que para ella eran tan sencillas. A Santi, quien se sumó a nuestro trabajo e incontables veces me ayudó a avanzar. Son los mejores.

A mi familia. A mi mamá, por ser una fuente inagotable de inspiración, una mujer increíble que se ha reinventado una y otra vez y siempre de forma maravillosa. A Manu, por ser la mejor hermana que podría pedir. Se que no importa lo que pase siempre puedo contar con tu apoyo. A Cande, se que no podemos ser mas distintos, pero me alegra decir cuando hace falta nos cubrimos la espalda. Seguí así que la rompes. A mi tío: jamás podría explicarle lo que hizo por mí en estos años, y mucho menos pagárselo. Es de las mejores personas que tuve el placer de conocer. A mis abuelos y a mi viejo.

Gracias a mis amigos: Nico, Tobi, Alex, que a día de hoy siguen sin entender qué hago. Qué felicidad encontrarnos, y qué felicidad que ni la distancia ni el tiempo deshagan la alegría del encuentro. Nico, en particular, verdaderamente sos un hermano para mí. Gracias por tanto, loco.

A la gente de la facu, que muy para su pesar tiene que aguantarme todos los días. Gracias Lu, Juli, Juanchi, Vicky, Iara, Santi (el Gringo), Vicky, Clari, Gonza, Nico, Vicky, Pedrito, Rezza, a Bruno por la paciencia y a Lean por ser inimitable. No creo que pueda resumir todo lo que pasamos en palabras, y tampoco me animaría a intentar. Les dedico todo.

Hay tanta gente que me faltó mencionar. Me encantaría poder dedicar hojas y hojas a toda la gente que estuvo ahí para mí, hablar de gente increíble que quien lea esto debería conocer, pero es el precio de tanto amor acumulado lo que hace imposible encerrarlo en pocas palabras. La gente que me acompañó a jugar al fútbol, la que vivió conmigo, la que me cebó mates al borde de la crisis, la que bailó conmigo, la que tocó música conmigo, la que me manda memes constantemente: a toda esa gente, un millón de gracias.

Muchas gracias al tribunal examinador **** y ****, por su buena predisposición y sus aportes, que permitieron hacer de este un trabajo mejor.

Gracias a la Universidad Nacional de La Plata, al lugar que me brindo de forma gratuita. Gracias al Obser, por existir. Gracias a la Astronomía. Gracias a todos.


\endinput
  %Agradecimientos
\clearemptydoublepage
\tableofcontents
\begin{Large}\textbf{Acr\'onimos}\end{Large}

\bigskip

Lista de acr\'onimos utilizados en esta tesis (notar que las siglas usualmente corresponden a las utilizadas en el idioma ingl\'es):

\begin{itemize}

 
 \item AGN:Núcleos galácticos activos (\textit{Active Galactic Nuclei})
 \item SMBH: Agujeros negros supermasivos (\textit{Supermassive Black Holes})
 \item LINER: Regiones nucleares emisoras de baja ionización (\textit{Low-ionization nuclear emission-line region})
 \item BLR: Región de líneas anchas (\textit{Broad Line Region})
 \item NLR: Región de líneas estrechas (\textit{Narrow Line Region})
 \item FR~I: Fanaroff–Riley tipo I
 \item FR~II: Fanaroff–Riley tipo II
 \item BZ: Blandford-Znajek
 \item BP: Blandford-Payne
 \item IC: Compton inverso (\textit{Inverse Compton})
 \item SSC: \textit{Synchrotron Self Compton}
 \item DSA: Aceleraci\'on difusiva por choques (\textit{Diffusive Shock Acceleration})
 \item SSA: Auto-absorci\'on sincrotrón (\textit{Synchrotron Self Absorption})
 \item EHT: Telescopio \textit{Event Horizont} (\texit{Event Horizont Telescope})
 \item VLBI: Interferometria de muy larga base (\textit{Very Long Baseline Interferometry})
 \item PSF: Función de dispersión de punto (\textit{Point Spread Function}
 \item FWHM: Anchura a media altura (\textit{Full Width at Half Medium}) 
 \item SED: Distribución espectral de energía (\textit{Spectral Energy Distribution})
 
 
 
 
 
 
 


\end{itemize}
  % Acrónimos
\clearemptydoublepage
\listoffigures
\clearemptydoublepage
\end{comment}

\mainmatter
%\chapter{Introducción}\label{cha:introduccion}

La presente Tesis se enmarca en trabajo realizado por el grupo... 
El propósito de este estudio es obtener una caracterización más detallada de...


%=====================================================
%
\section{Contexto}
%
%=====================================================

Contexto del trabajo... 


%=====================================================
%
\section{Objetivos}
%
%=====================================================
Este trabajo de tesis tiene por objetivo principal caracterizar... 

Adicionalmente, dentro de este trabajo se contempla:
\begin{itemize}
    \itemsep0em
    \item Caracterizar X cosa.
    \item Estimar Y parámetro.
    \item Estudiar el posible efecto de Z en coso.
\end{itemize}

%=====================================================
%
\section{Metodología}
%
%=====================================================

En este trabajo se llevó a cabo un análisis detallado de un conjunto de más de un centenar de observaciones realizadas con...
Exploramos los efectos físicos... 

La metodología de trabajo consistió en estudiar ... 

De esta manera también pondremos a prueba hipótesis tales como la existencia de...

%Para ello, utilizamos códigos públicos desarrollados por miembros de la comunidad internacional, tales como \texttt{PRESTO} \citep{Ransom2011} y otros de elaboración propia (\url{https://github.com/lalala/sarasa}).

\bigskip
\bigskip

\noindent La tesis está organizada de la siguiente manera:
%
\begin{itemize}
    \item En el Cap.~\ref{cha:radio} se introducen las principales definiciones y conceptos elementales de...
    Detallamos las características técnicas del instrumental utilizado...
    \item En el Cap.~\ref{cha:ondas} presentamos los resultados obtenidos mediante un análisis de... 
    \item Finalmente, en el Cap.~\ref{cha:conclusiones} presentamos las conclusiones y las perspectivas a futuro de esta línea de investigación.
\end{itemize}

%\clearemptydoublepage

%\chapter{Marco Te\'orico}\label{cha:Marco}

En este capítulo introducimos algunos de los conceptos teóricos clave para entender el desarrollo de nuestro modelo de \jet con estructura. En primera instancia, realizamos una descripción básica de los agujeros negros, así como los mecanismos de lanzamiento de \jets, para continuar con una explicación detallada de las propiedades de los \jets típicos de las radiogalaxias. Seguidamente, describimos brevemente la fenomenología, el descubrimiento y la clasificaci\'on detrás de los núcleos galácticos activos (AGN, del inglés \textit{Active Galactic Nuclei}) y en particular del subgrupo de las radiogalaxias. Hacia el final del capítulo, describimos brevemente las propiedades de Centaurus A, la radiogalaxia más cercana a la Vía Láctea y cuya emisión modelamos en el desarrollo de este trabajo.
%=====================================================
%


\section{Agujeros Negros}

\subsection{Descripci\'on básica}

En el contexto de la teoría de la relatividad general formulada por Einstein, la gravedad resulta de la manifestación geométrica de un espaciotiempo curvado. A su vez, dicha curvatura del espaciotiempo queda determinada por el contenido de energía e impulso de la materia en el mismo a través de la relación dada por las ecuaciones de campo de Einstein. Formalmente, un agujero negro es una región del espaciotiempo causalmente desconectada del resto. Esta región está encerrada por una hipersuperficie nula llamada horizonte de eventos; cualquier partícula que cruce el horizonte en dirección al agujero negro no puede volver a cruzarlo hacia el exterior.

\subsection{Soluciones de agujero negro.}\label{ssec:AN}

\subsubsection*{Solución de Schwarzschild.}
La solución de Schwarzschild \citep{Schwarzschild1916} es una solución de las ecuaciones de campo de Einstein que representa un espaciotiempo vacío, esféricamente simétrico, estacionario y estático alrededor de un cuerpo de masa M. Dicha métrica se expresa como
\[
ds^2 = -\left(1-\frac{2GM}{c^2r}\right)c^2 dt^2 + \left(1-\frac{2GM}{c^2r}\right)^{-1} dr^2 + r^2 d\Omega^2,
\]
donde $r_s = 2GM/c^2$ define el \textit{radio de Schwarzschild}, correspondiente al horizonte de eventos. En este radio, la componente temporal de la métrica se anula, lo que implica que ningún observador o señal puede escapar al infinito una vez que cruza dicha superficie. Adicionalmente, espaciotiempo de Schwarzschild posee una singularidad física en $r=0$, donde las cantidades de curvatura divergen. 
Esta solución describe la curvatura del espaciotiempo en la región externa a un agujero negro no rotante y sin carga eléctrica. La solución de Schwarzschild es la solución de agujero negro más simple posible, ya que está caracterizada por un único parámetro, la masa $M$ del objeto central (e.g., \citealt{Carroll2004}).

\subsubsection*{Soluci\'on de Kerr.}

La solución de Kerr \citep{Kerr1963} constituye la generalización de la métrica de Schwarzschild para el caso de un agujero negro en rotación. Fue hallada por Roy P. Kerr en 1963 y describe un espaciotiempo estacionario, axisimétrico y asintóticamente plano en torno a un cuerpo con momento angular distinto de cero. En este caso el elemento de linea en coordenadas de Boyer–Lindquist se escribe como
\begin{equation*}
	ds^2 = -\left(1-\frac{2GMr}{\Sigma c^2}\right)c^2dt^2 - \frac{4GMar\sin^2\theta}{\Sigma c^3}c\,dt\,d\phi + \frac{\Sigma}{\Delta}dr^2 
\end{equation*}
\begin{equation*}
	+ \Sigma d\theta^2 + \left(r^2 + a^2 + \frac{2GMa^2r\sin^2\theta}{\Sigma c^2}\right)\sin^2\theta\,d\phi^2,
\end{equation*}
donde $a = J/Mc$ es el parámetro de rotación, $\Sigma = r^2 + a^2\cos^2\theta$ y $\Delta = r^2 - 2GMr/c^2 + a^2$. 

Esta solución presenta dos horizontes (exterior e interior) definidos por $\Delta=0$, y una {ergósfera}, región comprendida entre el horizonte exterior y el {límite estático}, en la cual ningún observador puede permanecer en reposo respecto al infinito debido al arrastre del espacio-tiempo (\textit{frame dragging}) inducido por la rotación del agujero negro. La singularidad en este caso no es puntual sino en forma de anillo, localizada en $r=0$ y $\theta=\pi/2$.

Los agujeros negros descritos por esta solución se consideran el modelo más general de agujero negro astrofísico, ya que poseen masa y momento angular. No se espera que los agujeros negros reales presenten carga eléctrica significativa, por lo que la métrica de Kerr representa una descripción adecuada para la mayoría de los agujeros negros observados en el Universo (e.g., \citealt{Carroll2004}). Los agujeros negros rotantes son parte clave del proceso de lanzamiento de \jets relativistas, como presentaremos m\'as adelante.

\textcolor{red}{agregar algun diagrama.}

\section{Jets} \label{sec:Jets}

Los \jets son flujos altamente colimados de partículas y campos electromagnéticos. Suelen presentarse en una gran variedad de fuentes astrofísicas y caracterizarse por su gran extensión, la cual suele exceder el tamaño de las fuentes compactas que las producen por varios órdenes de magnitud. A su vez, su emisión puede cubrir un gran rango del espectro electromagnético.

En los AGNs, los \jets son impulsados por el SMBH central, poseen ángulos de apertura de unos pocos grados y, dependiendo de su potencia así como de las propiedades del medio que atraviesan, pueden propagarse hasta escalas de kiloparsecs o incluso megaparsecs \citep{Boccardi2017}. El plasma que conforma el \jet a su vez arrastra campos magnéticosen su interior, así como partículas cargadas relativistas. La interacción entre estas partículas y los campos magnéticos da lugar a la emisión sincrotrón observada principalmente en radio. 

\begin{comment}
\subsection{Mecanismo de Blandford-Znajek \textcolor{red}{Tipo al final de jets? Despues de aceleracion?}}\label{ssec:MecBZ}

Para un agujero negro rotante (AN de Kerr) de masa $M$ y parámetro de spin $a$ puede extraerse una cantidad de energía 

\begin{equation}
    E=Mc^2\left\{1-\sqrt{\frac{1}{2}[1+(1-a^2)^{1/2}]}\right\}
\end{equation}

De estar inmerso en una magnetosfera, como aquella sostenida por el flujo de acreción en las cercanías de un agujero negro, puede transferirse energía rotacional al campo magnético, generando una estructura helicoidal que se expande verticalmente produciendo un flujo de Poynting . Este mecanismo se denomina mecanismo de Blandford-Znajek (Blandford \& Znajek 1977).
La rotaci\'on  de las l\'ines de campo determina el cilindro de luz, donde la velocidad de rotaci\'on alcanza la velocidad de la luz y los efectos relativistas se vuelven significativos, a una distancia $R_{\text{LC}} \sim c/\Omega_L \sim c/(0.5\Omega_H)$ en estado estacionario, con $\Omega_L$ la velocidad angular de rotación de las lineas de campo magnético y $\Omega_H$ la velocidad angular de rotaci\'on en el horizonte. Luego, para valores del espín del agujero negro $a\le 0.95$, la potencia liberada resulta (Tchekhovskoy et al., 2010)
\begin{equation}
   P_{\text{BZ}}=\frac{\kappa}{4\pi c}\Phi_{\text{AN}}^2\Omega_{H}^2f(\Omega_H),
\end{equation}
donde $\kappa\approx 0.05$ y 

\begin{equation}
     f(\Omega_H) = 1 + 1.38 \left( \frac{\Omega_H R_g}{c} \right)^2 - 9.2 \left( \frac{\Omega_H R_g}{c} \right)^4.
\end{equation}
\end{comment}

%
%
%=====================================================%
\subsection{Estructura del \jet} \textcolor{red}{Mejor espero los resultados}\label{ssec:estructura}%(Boccardi et al. 2017)

Como mencionamos anteriormente, el mecanismo de lanzamiento y colimación de los \jets se considera de origen electromagnético. El campo magnético necesario para mantener la colimación y acelerar partículas se genera en las cercanías del horizonte de eventos, alimentado por el proceso de acreción. Los \jets de AGN más poderosos pueden permanecer colimados hasta escalas de cientos de pársecs, terminando en regiones de alta emisión denominadas \textit{hot spots}, mientras que los menos energéticos tienden a disiparse más rápidamente, adoptando morfologías difusas o en forma de ``plumas''.  

La geometría de los \textit{jets} puede describirse, en promedio, como autosimilar: una región inicial parabólica que transiciona a una expansión cónica en grandes escalas. No obstante, debido a efectos locales, los \jets presentan una gran diversidad de subestructuras, incluyendo choques internos, curvaturas, turbulencia e inestabilidades originadas por la interacción con el medio circundante.  

En cuanto a los efectos relativistas, consideremos un fotón emitido por un \jets que se desplaza con velocidad relativista hacia afuera desde un AGN. Si en el marco del \jet el fotón es emitido en dirección perpendicular al movimiento ($\theta' = 90^\circ$), en el marco del observador se detecta viajando con un ángulo

\begin{equation}\label{eq:marco-1}
    \sin\theta = \frac{1}{\Gamma},
\end{equation}
donde $\Gamma$ es el factor de Lorentz asociado al movimiento relativista del \jet. Este efecto, conocido como \textit{beaming}, concentra la radiación en la dirección de avance del flujo (ver \ref{sec:Beaming}). En el modelo estándar, los AGN dominados por el núcleo (\textit{core-dominated AGN}) muestran una emisión fuertemente \textit{beameada}, lo que indica que sus \jet están orientados cerca de nuestra línea de visión. Por otro lado, la alta velocidad del plasma implica que la radiación emitida está sujeta a un corrimiento Doppler, modificando las frecuencias observadas respecto a las emitidas.

El flujo observado difiere del flujo intrínseco por un factor proporcional a $\delta^3$, donde $\delta$ es el factor Doppler. Este fenómeno explica por qué resulta difícil detectar los contrajets en AGN dominados por la emisión nuclear: la emisión proveniente de la componente viajando en dirección opuesta al observador se ve fuertemente atenuada por efectos relativistas.


En el caso de algunos AGN cercanos se ha detectado que sus \textit{jets} presentan bordes brillantes (\textit{limb brightening}); es decir, se observa un aumento del brillo hacia las regiones periféricas del flujo. Este fenómeno ha sido reportado en diversos estudios, por ejemplo en M87 (\citealt{Mertens2016,Walker2016}), 3C 87 \citep{Nagai2019} y Cygnus A \citep{Boccardi2015} y suele interpretarse como evidencia de una diferencia de velocidades entre el \textit{spine} central y la envoltura, lo que refuerza los modelos de \textit{jets} estratificados tanto en velocidad como en estructura magnética. Sin embargo, se han propuesto varias explicaciones para el fenómeno de bordes brillantes, incluyendo la mencionada diferencia de velocidad entre la región central del \jet (\textit{spine}) y una envoltura más lenta que lo rodea (\citealt[e.g.][]{Mertens2016,Walker2016,Nagai2019}), cuya emisión estaría preferencialmente \textit{beameada} hacia el observador (ver Sec. \ref{sec:Beaming}); la carga de masa en los bordes del \textit{jet} \citep[ver ][]{Bruni2021}; reconexión magnética y aceleración de partículas en las regiones periféricas del \jet \citep[e.g.][]{Yang2024}; y un campo helicoidal o toroidal transportado por el \jet (este \'ultimo basado en que la emisividad sincrotrón es máxima cuando el campo magnético proyectado se encuentra en el plano del cielo, lo cual ocurre en los bordes del \textit{jet}) \citep[ver ][]{Nakamura2018}.


\subsection{Lanzamiento de \jets: Mecanismos Blandford-Znajek y Blandford-Payne.}

La física de los \jets se relaciona con la extracción de energía tanto del agujero negro rotante (ver Sec.\ref{ssec:AN}), a través del mecanismo de Blandford-Znajek (BZ, \citealt{Blandford1977}) como del disco de acreción, mediante el mecanismo de Blandford-Payne (BP, \citealt{Blandford1982}). El campo magn\'etico juega un rol esencial: no solo canaliza la energ\'ia extra\'ida en forma de \jets, sino que tambi\'en contribuye a su colimaci\'on junto con la presi\'on externa del medio circundante. Existen dos modelos fundamentales que explican este fenómeno, según el origen de la rotación de las líneas de campo magnético:

\begin{itemize}
\item {Mecanismo de Blandford–Znajek (BZ, \citealt{Blandford1977}):} en este caso, la rotación de las líneas de campo magnético no es impuesta por el disco, sino que surge del arrastre de marcos de referencia en la vecindad de un agujero negro en rotación. El campo magnético, sostenido por el disco de acreción, penetra el horizonte de eventos y, al interactuar con el espacio-tiempo en rotación, induce un flujo de energía electromagnética que permite extraer energía directamente de la rotación del agujero negro.
\item {Mecanismo de Blandford–Payne (BP, \citealt{Blandford1982}):} aquí, el campo magnético está “congelado” en el plasma del disco de acreción. A medida que el disco rota, las líneas de campo magnético también lo hacen. El plasma puede ser acelerado centrifugamente hacia el exterior, escapando y dando lugar al \textit{jet}. Este proceso se caracteriza por generar \jets menos colimados, con velocidades moderadamente relativistas y potencias menores que en el caso BZ.
\end{itemize}

En la práctica, ambos procesos pueden coexistir: mientras que el mecanismo BP facilita la eyección inicial de material desde el disco, el mecanismo BZ suele dominar en la producción de los \jets más energéticos. Así, el disco de acreción aporta el campo magnético necesario, mientras que la rotación del agujero negro suministra gran parte de la energía del \jet.
\begin{comment}
\textcolor{blue}{En algun lado hay que mencionar cualitativamente como se aceleran y coliman los jets y por que se espera que haya una diferencia entre el sheath y el spine}
\end{comment}
%La estructura de la emisión en radio de un AGN por observaciones VLBI resulta generalmente en un núcleo brillante y, generalmente, no resuelto, junto con una estructura de \jet que emana de dicho núcleo. Esta morfología se interpreta como consecuencia de efectos de selección asociados tanto a la naturaleza relativista del \jet como a la sensibilidad del arreglo interferométrico.
%Dada la velocidad relativista del fluido que compone el jet, su radiación está sujeta a un corrimiento Doppler. En consecuencia, el flujo observado y el flujo emitido del \jet difieren por un factor $\delta^3$ con $\alpha$ el indice espectral, $n$ un parámetro de ajuste y $\delta$ es el factor de Doppler \textcolor{red}{Un efecto es el corrimiento doppler que me corre las frecuencias y otro es el beaming que me arruina la isotropia de la radiacion original >:c}

%\begin{equation}
%    \delta=\frac{1}{\Gamma(1-\beta\cos(\theta))},
%\end{equation}
%donde $\beta=v/c$ es la velocidad del flujo normalizada respecto de la velocidad de la luz, $\Gamma=1/\sqrt{1-\beta^2}$ el factor de Lorentz y $\theta$ el \'angulo  entre la direcci\'on de propagaci\'on y la linea de la visi\'on %(Boccardi et al. 2017)

%----------------------------------------------------------
%
\section{Emisi\'on en el \jet}

El modelo estándar del espectro de energía de un \jet relativista presenta típicamente dos componentes con forma de ``doble joroba''. El primer pico, a bajas energías (radio–óptico–rayos X blandos), se asocia a la radiación sincrotrón producida por electrones relativistas que se mueven en el campo magnético del \jet. El segundo pico, ubicado a energías más altas (rayos X duros–gamma), se interpreta como resultado de procesos de dispersión Compton inversa (IC, por sus siglas en inglés), en los cuales los mismos electrones relativistas transfieren energía a fotones de menor energía (e.g., \citealt{1995PASP..107..803U, Ulrich1997}). Estos ``fotones semilla”  pueden provenir de la propia emisión sincrotrón, que interactúa con los electrones por un proceso conocido como \textit{Synchrotron Self Compton} (SSC). También puede ocurrir interacción IC con fotones provenientes del flujo de acreción, la región de líneas anchas u otras regiones emisoras del AGN.

Además de los modelos {leptónicos}, en los cuales la radiación observada a lo largo del espectro electromagnético es producida principalmente por electrones y positrones relativistas (a la vez que los protones del flujo no alcanzan energías suficientes como para contribuir de manera significativa), existen también modelos {lepto-hadrónicos} que buscan reproducir la distribución espectral de energía observada en los \textit{jets}. En estos modelos, tanto los electrones primarios como los protones son acelerados hasta energías ultrarrelativistas. Los protones, en particular, pueden alcanzar el umbral energético necesario para la producción de fotopiones mediante interacciones con el campo de fotones suaves. En este marco, la emisión sincrotrón de los electrones primarios domina el espectro en las bajas frecuencias, mientras que en las altas frecuencias la emisión está gobernada por una combinación de procesos: sincrotrón de protones, decaimiento de piones neutros, emisión Compton y sincrotrón de los productos secundarios del decaimiento de piones cargados. Adicionalmente, los fotones de alta energía generados pueden ser absorbidos por interacciones fotón-fotón, dando origen a cascadas electromagnéticas que contribuyen de manera compleja al espectro total observado \citep{Böttcher2013}.


\begin{comment}
    
Dada una partícula cargada moviéndose en una regi\'on donde se encuentra un campo electromagnético, esta experimenta una fuerza de Lorentz

\begin{equation}
    \frac{d\Vec{p}}{dt}=q\left(\Vec{E}+\frac{\Vec{v}\times\Vec{B}}{c}\right),
\end{equation}
donde $q$ es la carga de la partícula, $\Vec{v}$ su velocidad y $\Vec{E},~\Vec{B}$ son los campos eléctrico y magnético en la región.

Debido a que la componente de la fuerza debido al campo magnético es perpendicular a la velocidad esta no modifica su modulo y en consecuencia la variación de la energía resulta

\begin{equation}
    \frac{d\Vec{E}}{dt}=q\frac{d\Vec{r}}{dt}\Vec{\nabla}V,
\end{equation}
con $V$ el potencial escalar. Luego, la forma más sencilla de acelerar una partícula cargada resulta someterla a una diferencia de potencial.
\end{comment}

\textcolor{blue}{Existen múltiples modelos para la aceleración de partículas en un \textit{jet}. Un ejemplo es la aceleración difusiva por shocks: en distintas regiones del \jet pueden desarrollarse inestabilidades hidrodinámicas y magnetohidrodinámicas que generan choques internos. En estos choques, las partículas relativistas pueden ser aceleradas mediante procesos de aceleración Fermi tipo I (\textit{Diffusive Shock Acceleration} o DSA), al cruzar repetidamente la discontinuidad de choque \citep[e.g. ]{Drury1983}. En un modelo magnetizado el plasma tiende a ser menos compresible, lo que reduce la eficiencia de la aceleración por choques \citep{Romero2018}. 

Alternativamente, puede considerarse la reconexión  magnética, donde lineas de campo en apuntando en direcciones opuestas pueden ser forzadas a unirse producto de inestabilidades (e.g., \citealt{Begelman1998}, \citealt{Barniol2017}), dando lugar a reconfiguraciones de las líneas de campo magnético liberan energía que puede transferirse a las partículas.}\textcolor{red}{expandir}


\begin{comment}

\textcolor{red}{Una pequeña introducción de que voy a tratar y nada muuuy profundo, algo de reconexi\'on magnética pero tampoco es una hipótesis necesaria del modelo, simplemente asumimos que se aceleran partículas. Lo puedo incluir en otra parte.}

\textcolor{blue}{En distintas partes del pueden tener inestabilidades, generar shocks y las partículas pueden acelerarse en estos shocks cruzando de un lado al otro. Para nuestro modelo magnetizado el plasma se hace mas incompresible por lo cual nada muy efectivo. Reconexi\'on magnética tampoco muy en detalle}
    
\end{comment}

\section{Beaming relativista}\label{sec:Beaming}

En sistemas astrofísicos donde las partículas emisoras se desplazan a velocidades relativistas ($v \sim c$), la radiación producida no es isotrópica en el marco del observador. Este fenómeno, conocido como \textit{beaming relativista} o \textit{Doppler boosting}, resulta de la transformación relativista de la intensidad y frecuencia del campo de radiación debido al movimiento del emisor.

Bajo estas correcciones, la frecuencia observada $\nu_{\text{obs}}$ y la intensidad $I_{\nu,\text{obs}}$ se relacionan con sus valores en el sistema de reposo del plasma ($\nu'$ e $I'_{\nu'}$) mediante el factor Doppler:

\begin{equation}\label{eq:marco-9}
\delta = \frac{1}{\Gamma(1 - \beta \cos\theta)}.
\end{equation}
donde $\Gamma = (1 - \beta^2)^{-1/2}$ el factor de Lorentz del flujo, con $\beta = v/c$, y $\theta$ el ángulo entre la dirección de movimiento y la línea de visión del observador

Bajo esta transformación, la frecuencia y la intensidad observadas son:

\begin{equation}\label{eq:marco-10}
\nu_{\text{obs}} = \delta\,\nu',
\end{equation}

\begin{equation}\label{eq:marco-11}
I_{\nu,\text{obs}} = \delta^3 I'_{\nu'}.
\end{equation}

La potencia observada de una fuente móvil, integrada en frecuencia, se ve modificada en un factor $\delta^4$ respecto a la potencia emitida en el marco del plasma. Este efecto puede producir una amplificación significativa del flujo aparente cuando la línea de visión se encuentra dentro del cono de apertura del haz relativista ($\theta \lesssim 1/\Gamma$).
%----------------------------------------------------------

%===========================================================================================================================
\section{Núcleos galácticos activos}\label{sec:AGNs}
%=====================================================

Los AGN son fenómenos asociados a agujeros negros supermasivos (SMBH, del inglés \textit{Supermassive Black Holes}) contenidos en el núcleo de ciertas galaxias. Estos agujeros negros, con masas del orden de $10^6$–$10^{10}  M_\odot$, pueden acretar gas de su entorno por medio de un flujo de acreción. Durante este proceso, la energía gravitatoria del material en caída se convierte progresivamente en energía cinética y posteriormente en radiación, lo que permite que el sistema libere enormes cantidades de energía, a menudo superando la luminosidad total de la galaxia huésped. 
En un flujo de acreción, el esfuerzo viscoso (producto de turbulencias de origen magnetohidrodinámico) permite transferir momento angular hacia las regiones externas, causando que el material caiga en una trayectoria espiral hacia el SMBH. La pérdida de energía gravitatoria al acercarse al agujero negro se traduce en un aumento de la energía cinética orbital del gas ($E_\mathrm{grav} \to E_\mathrm{kin}$), que luego se disipa en forma de energía interna del fluido por efecto de la viscosidad turbulenta ($E_\mathrm{kin} \to E_\mathrm{th}$), calentando el plasma. Finalmente, el gas caliente emite radiación electromagnética (principalmente en los rangos óptico, ultravioleta y de rayos X), liberando eficientemente la energía acumulada ($E_\mathrm{th} \to E_\mathrm{rad}$). 
De este modo, el disco de acreción actúa como un mecanismo disipativo extremadamente eficiente, capaz de convertir hasta $\sim 10\%$ de la energía en reposo de la materia en radiación observable.


Si bien la primera observación de núcleos galácticos con espectros inusuales se atribuye a \cite{Fath1909}, la idea de un “motor central” fue anticipada por Carl Seyfert (\citeyear{Seyfert1943}), quien identificó un conjunto de galaxias con núcleos que presentaban líneas de emisión muy anchas, interpretadas como gas ionizado orbitando a gran velocidad en la vecindad del pozo gravitatorio central. 
Otro de los hallazgos clave fue la detección de \jets relativistas: chorros colimados de plasma que se extienden a escalas de kiloparsecs o incluso megaparsecs. Una de las primeras observaciones de \jets corresponde a la galaxia M87, una galaxia elíptica gigante que se encuentra en la zona norte del cúmulo de Virgo, a una distancia de $\sim16.1$ Mpc de la Vía Láctea; dicha observación fue realizada por \cite{Curtis1918}. Un ejemplo histórico es Cygnus A, una de las primeras radiogalaxias identificadas, cuya intensa emisión en radio reveló una estructura de doble lóbulo asociada a la eyección de plasma desde el núcleo \citep{Jennison1953}. Más tarde, el descubrimiento del cuásar 3C 273 (\citealt{Hazard1963}; \citealt{Schmidt1963}) daría lugar a otro gran avance, al mostrar un espectro con líneas de emisión anchas que más tarde se descubrirían que correspondían a líneas del hidrógeno fuertemente desplazadas al rojo, confirmando la naturaleza extragaláctica de estos objetos extremadamente luminosos.

La clasificación de los AGN puede abordarse desde distintos criterios (ej. ver \citealt{1995PASP..107..803U}):

\begin{itemize}
    \item Según la tasa de acreción y la luminosidad nuclear: los cuásares ópticos y radiocuásares se asocian con tasas de acreción altas (cercanas o superiores al límite de Eddington), mientras que radiogalaxias y LINER (\textit{Low-ionization nuclear emission-line region}) corresponden a tasas de acreción subcríticas.

\item Según la presencia de \jets: distinguimos AGN luminosos o débiles en radio (\textit{radio-loud/quiet}). Los primeros presentan \jets prominentes, mientras que los segundos no poseen \jets desarrollados.

\item Según la orientación: cuando el \jet apunta hacia la línea de visión del observador, se observa un blazar, caracterizado por variabilidad rápida, fuerte emisión no térmica y \jets aparentemente unilaterales debido a efectos relativistas que discutiremos mas adelante. Por el contrario, vistas en ángulos mayores, las mismas fuentes pueden clasificarse como radiogalaxias o cuásares.
\end{itemize}

Posteriormente, con el desarrollo del modelo unificado de AGN, se comprendió que estos objetos presentan una estructura mucho más compleja: un SMBH rodeado por un disco de acreción, a su vez rodeado por un toro de polvo y gas molecular. En las cercanías del disco se localiza la región de líneas anchas (\textit{Broad Line Region}, BLR), compuesta por nubes de gas ionizado que orbitan a velocidades de varios miles de kilómetros por segundo. A escalas mayores, más allá del toro, se encuentra la región de líneas estrechas (\textit{Narrow Line Region}, NLR), formada por gas menos denso y más frío, con velocidades de unos pocos cientos de kilómetros por segundo \citep{1995PASP..107..803U}. 

La geometría del sistema, junto con la tasa de acreción, determina gran parte de las diferencias observacionales entre las diversas clases de AGN. En conjunto, los AGN representan laboratorios naturales para estudiar la acreción, la dinámica de plasmas relativistas y la interacción entre los SMBH y las galaxias que los contienen, constituyendo algunos de los objetos más energéticos y complejos del universo.


\subsection{Radiogalaxias}
Las radiogalaxias son un tipo de AGN que presentan una morfología bi-lobular en radio. En muchos casos, también presentan un núcleo compacto que puede resolverse en el óptico o en otras bandas. Los lóbulos de radio son alimentados continuamente por los \textit{jets} relativistas que emergen desde el núcleo activo y que transportan plasma a grandes distancias (e.g., \citealt{Fanaroff1974, begelman1984theory, Blandford2019}).

\begin{figure}[h!]
    \centering
    \includegraphics[width=0.5\linewidth]{graphics/Marco/cenA_w22.jpg}
    \caption{Imagen en Rayos X de la radiogalaxia Centaurus A, Chandra: NASA/CXC/SAO}
    \label{fig:mod-1.5}
\end{figure}

La emisión en radio se debe al proceso de {radiación sincrotrón}, producido cuando electrones relativistas (acelerados en el entorno del agujero negro y a lo largo del \jet) se mueven en trayectorias helicoidales alrededor de las líneas de campo magnético. Este mecanismo no térmico genera un espectro continuo con fuerte polarización, característico de las fuentes de radio extendidas.

Este tipo de fuentes se clasifican en dos categorías principales, conocidas como Fanaroff–Riley tipo I (FR~I) y tipo II (FR~II) \citep{Fanaroff1974}. Las galaxias FR~I presentan una morfología en la que las regiones de mayor brillo se localizan en las cercanías del núcleo y decaen hacia los extremos de los lóbulos. Los \jets en este tipo de fuentes tienden a ser más anchos, menos colimados y a perder energía de manera significativa en escalas relativamente cortas debido a su interacción con el medio circundante. Como consecuencia, la emisión es predominantemente difusa en los lóbulos externos.
En contraste, las galaxias FR~II presentan una morfología de borde brillante, caracterizada por la presencia de regiones de emisión intensa en los extremos de los lóbulos, asociadas a choques de terminación (\textit{hotspots}) producidos cuando los \jets relativistas, altamente colimados, impactan contra el medio intergaláctico. Estos \jets se mantienen estables y energéticamente dominantes hasta distancias muy grandes respecto al núcleo, dando lugar a estructuras extensas y bien definidas. Las FR~II corresponden a galaxias de radio de alta potencia.


En el caso de las galaxias FR I, estudios de emisión en rayos X \citep{Laing2007} sugieren que la aceleración de partículas no está confinada únicamente a regiones localizadas, sino que se encuentra distribuida a lo largo de todo el volumen del \jet. Esto refuerza la idea de que los \jets son sitios de aceleración eficiente de partículas relativistas a gran escala.

\begin{figure}[h]
\centering
\begin{subfigure}[b]{0.45\textwidth}
\centering
\includegraphics[width=\linewidth]{graphics/Marco/3C31-crop.pdf}
\end{subfigure}
\hfill
\begin{subfigure}[b]{0.45\textwidth}
\centering
\includegraphics[width=\linewidth]{graphics/Marco/3C98-crop.pdf}
\end{subfigure}
\caption[Imagen comparativa de galaxias FR~I y tipo II]{Imagenes de las galaxias 3C31 (izquierda), clasificada como FR tipo I, y 3C98 (derecha) clasificada como FR~II \citep{Hardcastle2020}}
\label{fig:galaxias_FR}
\end{figure}

%------------------------------------------

\section{Centaurus A}\label{ssec:CenA}

Centaurus A (PKS 1322-428, NGC 5128) es una galaxia elíptica ubicada en la constelación de Centaurus, en el hemisferio sur celeste, a una distancia de $\sim 3.8$ Mpc de la Vía Láctea \citep{Harris2010}. Se trata de la radiogalaxia más cercana y, por lo tanto, un laboratorio excepcional para estudiar en detalle los procesos físicos que ocurren en núcleos activos de galaxias.

Centaurus A alberga en su centro un SMBH con una masa estimada de $M = (5.5 \pm 3.0) \times 10^7 \ M_\odot$ (\citealt{Israel1998}; \citealt{Neumayer2010}). Desde el núcleo emergen \jets relativistas que se observan en radio y rayos X, extendiéndose desde escalas subparsec hasta cientos de kiloparsecs (\citealt{Clarke1992}; \citealt{Hardcastle2003}; \cite{Feain2011}). Su morfología general es consistente con una radiogalaxia de tipo FR I \citep{Fanaroff1974}, con un \jet bidireccional colimado hasta escalas de kiloparsec, que finalmente termina en lóbulos de radio brillantes. Estos lóbulos gigantes, que se extienden varios grados en el cielo (correspondientes a $\sim 600$ kpc en proyección), dominan la emisión en radio de la fuente y constituyen una de las estructuras más grandes observables asociadas a un AGN cercano.

Gracias a su cercanía, Centaurus A ha sido estudiada a lo largo de todo el espectro electromagnético. En particular, observaciones con técnicas de interferometría de muy larga base (VLBI) revelan la estructura interna del \jet en escalas de decenas de milisegundos de arco ($\sim 0.018$ pc en proyección). El programa TANAMI \citep{Mueller2011} ha monitoreado la fuente en frecuencias $8.4~\mathrm{GHz}$ y $22.3~\mathrm{GHz}$ correspondientes a la banda de radio, mostrando un \jet altamente colimado a distancias de pocos días luz del SMBH (Fig. \ref{fig:marco-1}).

\begin{figure}[h!]
    \centering
    \includegraphics[width=\linewidth]{graphics/Marco/TANAMI_CEN_A.png}
    \caption[Mapas de contorno de Cen A obtenido por el programa de monitoreo TANAMI]{Mapas de contorno de Cen A obtenido por el programa de monitoreo TANAMI en frecuencias de $8.4~\mathrm{GHz}$ y $22.3~\mathrm{GHz}$ \citep{Mueller2011}}
    \label{fig:marco-1}
\end{figure}

Recientemente, observaciones del Telescopio Horizonte de Eventos (EHT, por sus siglas en inglés) han permitido obtener imágenes de Centaurus A con resolución nominal de 25 microarcosegundos ($\mu$as) a longitudes de onda de 1.3 mm. Dicha resolución permite discriminar estructuras a escalas por debajo de 200 radios gravitacionales \textcolor{red}{Referencia}. 

Con este nivel de detalle se revela una estructura asimétrica altamente colimada, consistente con un \jet de borde brillante. Además, se detecta la presencia de un contra\jet más débil, lo que constituye un objetivo fundamental para el modelo desarrollado en este trabajo (\citealt{2021NatAs...5.1017J}; ver Fig.~\ref{fig:marco-2}).

\begin{figure}[h!]
    \centering
    \includegraphics[width=0.8\linewidth]{graphics/Marco/EHT_CEN_A.png}
    \caption[Estructura del \jet de Cen A. ]{Estructura del \jet de Cen A. \textbf{Izq.} Imagen obtenida del monitoreo TANAMI a una frecuencia de $8.4~\mathrm{GHz}$. Se muestra la temperatura de brillo en escala logarítmica (Noviembre 2011). \textbf{Der.} Imagen obtenida por el EHT a una frecuencia de $228~\mathrm{GHz}$ que revela estructuras a una escala mucho más pequeña. La escala de colores corresponde a la raíz cuadrada de la temperatura de brillo (Abril 2017).}
    \label{fig:marco-2}
\end{figure}


Por estas características, Centaurus A es un caso de estudio clave para comprender cómo los \jets extraen energía del SMBH y se propagan a través del medio intergaláctico. En este trabajo, utilizamos observaciones de alta resolución (VLBI) obtenidas por el EHT para estudiar la morfología del \jet en escalas muy próximas al núcleo. Estos datos proporcionan una base sólida para contrastar modelos de emisión que dan lugar a la estructura \text{espina-vaina} (\textit{spine–sheath}) observada.

\endinput

%\clearemptydoublepage
%\chapter{Modelo}\label{cha:modelo}

En este capítulo presentamos un modelo semianalítico de emisión para \jets con estructura multizona, desarrollado a partir de \textcolor{red}{cita}. En primer lugar, describimos la estructura general del modelo, incluyendo las escalas adoptadas para caracterizar la inyección de partículas. A continuación, analizamos la evolución de los parámetros geométricos y dinámicos del \textit{jet}, así como su impacto en la potencia total y en la redistribución de energía asociada a la inyección. Posteriormente, se introduce la ecuación de transporte relativista utilizada para describir la evolución de la distribución de electrones relativistas. Finalmente, se detalla la emisión sincrotrón y la expresión empleada para la profundidad óptica en función de los radios interno y externo del modelo.  

\begin{comment}
Para la comparación con datos observacionales se consideran todas las restricciones disponibles en la literatura sobre las condiciones físicas de Centaurus A (Cen~A).  
\end{comment}

\begin{figure}[h!]
    \centering
    \includegraphics[width=0.75\linewidth]{graphics/Modelo/conceptual_program_flow.png}
    \caption{Diagrama del flujo del código para modelo de \jet con estructura}
    \label{fig:mod-1}
\end{figure}

\section{Estructura general del \jet}
En nuestro modelo, postulamos un \jet relativista con estructura diferenciada en funci\'on del radio, distinguiendo la espina (\textit{spine}) para $r < R_\mathrm{in}$ \textcolor{red}{consideramos la espina vacia, explicar mejor esto} y la vaina/envoltura (\textit{sheath}) para $R_\mathrm{in} < r < R_\mathrm{out}$ (ver Fig. \ref{fig:mod-1.5}\textcolor{red}{Hacer un grafico, o encontrar alguno lindo}). El \jet se extiende desde una altura base $z_0$ hasta una distancia máxima $z_{\mathrm{max}} \gg z_0$. En las cercanías de la base, el \jet posee un radio externo inicial $R_0^{\mathrm{out}}$ y un radio interno $R_0^{\mathrm{in}}$. La inyección de partículas relativistas se implementa como un mecanismo continuo y coherente, consistente, por ejemplo, con procesos de reconexión magnética, y se lleva a cabo en la región $z_0 < z < z_{\mathrm{max,inj}}$. Para calcular la evolución de los distintos parámetros físicos, incluyendo la inyección de partículas, dividimos el \jet en segmentos sucesivos. En cada segmento se inyectan electrones, que luego son transportados a lo largo del flujo hacia segmentos más lejanos a medida que se enfrían.

%\begin{figure}[h!]
%    \centering
%    \includegraphics[width=0.75\linewidth]{graphics/Modelo/conceptual_program_flow.png}
%    \caption{Diagrama del flujo del código para modelo de \jet con estructura}
%    \label{fig:mod-1.5}
%\end{figure}

\subsection{Perdidas radiativas.}
Los electrones relativistas pierden energía a través de diferentes procesos, como radiación sincrotrón y pérdidas adiabáticas, asociadas al trabajo realizado por el plasma durante la expansión transversal del \textit{jet}. En el caso de la radiación sincrotrón, la perdida de energía por unidad de tiempo se obtiene directamente a partir de la expresión para la potencia 

\begin{equation}\label{eq:mod-1}
    -\frac{dE}{dt}\Bigg|_{\mathrm{syn}} = \frac{4}{3}\,c\,\sigma_{\mathrm{Th}}\,U_{\mathrm{mag}}\,\beta^2 \gamma^2,
\end{equation}
donde $U_{\mathrm{mag}} = B^2/8\pi$ es la densidad de energía magnética, $\beta = v/c$ es el cociente entre la velocidad de la partícula relativista y la velocidad de la luz, $\gamma$ es el factor de Lorentz del electrón, y $\sigma_{\mathrm{Th}} = \tfrac{8\pi}{3} r_e^2$ es la sección eficaz de Thomson, con $r_e$ el radio clásico del electrón.  

Sustituyendo $U_{\mathrm{mag}}$ y $\sigma_{\mathrm{Th}}$, y considerando $\beta \approx 1$ para electrones relativistas, se obtiene

\begin{equation}\label{eq:mod-2}
    -\frac{dE}{dt}\Bigg|_{\mathrm{syn}} = \frac{4}{9}\,c\,r_e^2\,B^2\,\gamma^2.
\end{equation}

\subsection{Perdidas adiabaticas}

Las pérdidas adiabáticas se refieren a la disminución de la energía interna de un sistema de plasma o gas cuando este se expande sin intercambio de calor con el entorno. Desde un punto de vista hidrodinámico, el término adiabático aparece naturalmente en la ecuación de energía del fluido y actúa incluso en ausencia de un medio externo. En un flujo en expansión, la energía interna se convierte parcialmente en energía cinética del movimiento colectivo del plasma, lo que produce una aceleración del flujo. En contraste, en un flujo de acreción, el proceso inverso conduce a un aumento de la energía interna a expensas del movimiento en masa, generando calentamiento compresional.

El flujo de energía de un \jet consiste predominantemente en la energía interna de los electrones y el movimiento de grupo dominado por iones. A estos componentes se suma una fracción significativa asociada al campo magnético, que puede transportar una parte no despreciable del flujo de energía total en forma de energía Poynting \citep{Blandford1977, Meier2001}. Al moverse en la dirección del \jet, los electrones pierden la parte isotrópica de su energía y ganan velocidad en dicha dirección. Los electrones forman un fluido magnetohidrodinámico junto con protones presentes y campos magnéticos. La energía perdida por procesos adiabáticos es convertida en energía cinética del \textit{jet}, es decir, la velocidad del \jet en estado estacionario aumenta con la distancia al n\'ucleo \citep{Zdziarski2014}. En nuestro modelo, este efecto se considera despreciable, dado que las variaciones en la energía cinética del \jet son pequeñas en comparación con las pérdidas radiativas locales.

La tasa de perdidas adiabaticas esta dada por

\begin{equation}\label{eq:mod-3}
 \frac{dE}{dt}\Bigg|_{\mathrm{Ad}}=-\frac{2}{3}\frac{d\ln(R_\mathrm{out}(z))}{dz}\frac{c\Gamma_\mathrm{j}\beta_\mathrm{j}(z)}{z}\left(E-\frac{(mc^2)^2}{E}\right), 
\end{equation}

Con E la energía en el marco de referencia del fluido.





\section{Geometría y dinámica del \jet}

Presentamos un \jet con geometría cuasi-parabólica multizona, cuya sección transversal está restringida por dos radios definidos como función de la altura $z$:

\begin{equation}\label{eq:mod-4}
    R^{\mathrm{out/in}}(z) = R^{\mathrm{out/in}}_0 \left(\frac{z}{z_0}\right)^w,
\end{equation}

donde $R^{\mathrm{out/in}}_0$ son los radios en la base del \jet y $w$ es el parámetro de apertura que determina la forma cuasi-parabólica del flujo. 
\begin{comment}
    En nuestro modelado de Centaurus A utilizamos $w = 0.33$ para ambos radios, en concordancia con las observaciones del EHT.

\end{comment}

Se espera que, al menos hasta las alturas máximas consideradas en nuestro modelo, el factor de Lorentz del \jet $\Gamma_\mathrm{j}$ siga una ley de potencias respecto de la distancia $z$:

\begin{equation}\label{eq:mod-5}
    \Gamma_\mathrm{j}(z) = \Gamma_{\mathrm{j},0} \left(\frac{z}{z_0}\right)^g,
\end{equation}
con $\Gamma_{\mathrm{j},0} \approx 1$, de modo que el \jet es casi no relativista en su base y se acelera progresivamente a medida que $z$ aumenta. El parámetro $g$ controla la tasa de aceleración: valores mayores implican una aceleración más rápida del flujo relativista cerca de la base del \jet.

\subsection{Ecuaciones de conservación y magnetizaci\'on}
Debido a la ecuación de continuidad relativista, $\nabla_{\alpha}(\rho' u^\alpha)=0$, con $\rho'$ la densidad en el marco comóvil y $u^\alpha$ la tetra velocidad del fluido, el caudal de masa del \textit{jet} $\dot{M}_{\mathrm{j}}$ es constante con $z$. De forma análoga, si consideramos la conservación del tensor de energía–impulso, $\nabla_\mu T^{\mu\nu}=0$, y despreciamos la radiación emitida localmente, la potencia total del \textit{jet} se conserva y puede escribirse como
\begin{equation}\label{eq:mod-6}
    L_{\mathrm{j}}=\int_\Sigma T^{0z}\, d\Sigma_z
    = \Gamma_{\mathrm{j}}\,\dot{M}_{\mathrm{j}}\,c^2\,h'\,(1+\sigma') \equiv \dot{M}_{\mathrm{j}} c^2\,\mu ,
\end{equation}
donde hemos definido el invariante $\mu \;\equiv\; \Gamma_{\mathrm{j}}\,h'\,(1+\sigma')$, $\sigma'$ es la magnetización
\begin{equation}\label{eq:mod-7}
    \sigma'=\frac{B'^2}{4\pi\,\rho' c^2 h'} ,
\end{equation}
y $h'$ la entalpía específica (adimensional),
\begin{equation}\label{eq:mod-8}
    h' \;=\; 1+\frac{u_e'+p_e'}{\rho' c^2} ,
\end{equation}
con $u_e'$ y $p_e'$ la energía interna y la presión (en el marco comóvil) respectivamente.

Igualando $\dot{M}_{\mathrm{j}} = \pi\,R(z)^2\,\Gamma_{\mathrm{j}}(z)\,\beta_{\mathrm{j}}(z)\,\rho' c$, se obtiene que el módulo del campo magnético en el marco del fluido que resulta
\begin{equation}\label{eq:mod-9}
    B'(z)=\frac{2}{\Gamma_{\mathrm{j}}(z)\,R(z)}\,
    \sqrt{\frac{L_{\mathrm{j}}}{c\,\beta_{\mathrm{j}}(z)}\left(\frac{\sigma'}{1+\sigma'}\right)},
\end{equation}
con
\begin{equation}\label{eq:mod-10}
    R(z)=\sqrt{[R^{\mathrm{out}}(z)]^2-[R^{\mathrm{in}}(z)]^2}
\end{equation}
el radio efectivo de la sección anular transversal a $z$.

De aquí se sigue qu\'e $B'(z)\propto [\Gamma_{\mathrm{j}}(z)\,R(z)]^{-1}$. De aqu\'i obtenemos que al expandirse y acelerarse el \textit{jet}, el campo se debilita por dilución y conversión de energía magnética en energía cinética.

De la Ec. \eqref{eq:mod-6} se deduce que la energía por unidad de flujo de masa, $\mu=L_{\mathrm{j}}/(\dot{M}_{\mathrm{j}} c^2)$, es constante. Cerca de la base, donde el plasma es aproximadamente frío ($h'_0 !\approx! 1$) y el flujo aún no ha adquirido velocidades relativistas, se tiene entonces
\[
    \mu \;\approx\; (1+\sigma'_0) \quad \Rightarrow\quad \sigma'_0 \;\approx\; \mu-1 .
\]
Si el \jet alcanza factores de Lorentz apreciables más adelante (y/o transporta gran potencia con un caudal de masa moderado), necesariamente $\mu\gg1$ y, por tanto, $\sigma'_0\gg1$: al inicio la energía está mayormente en el campo. La aceleración consiste precisamente en convertir energía magnética en cinética, haciendo que $\sigma'$ decrezca mientras $\Gamma_{\mathrm{j}}$ aumenta, manteniendo $\mu=\Gamma_{\mathrm{j}} h' (1+\sigma')$ constante.



\section{Potencia del \jet e inyección de partículas}

En esta sección exploramos en detalle los distintos componentes que contribuyen a la potencia total de un \jet relativista, así como los mecanismos responsables de la inyección y aceleración de partículas dentro del flujo. La potencia del \jet constituye un parámetro fundamental para caracterizar la eficiencia del proceso de extracción de energía desde el entorno del agujero negro y su posterior transporte a escalas macroscópicas. Dependiendo de la composición del plasma, la fracción de energía cinética, magnética y radiativa puede variar significativamente, afectando tanto la evolución dinámica del \jet como su emisión observada en distintas bandas del espectro electromagnético.

\subsection{Potencia total del \jet}

Los \jets pueden estar compuestos por un plasma frío de protones y electrones, sumado a una pequeña contribución de partículas relativistas, o por un plasma puramente leptónico de electrones y positrones relativistas \citep{Romero2017}. En general, la potencia de un \jet a una cierta altura desde su base puede separarse en sus distintas componentes: la potencia transportada por "materia fría", que corresponde a protones y electrones asociados no relativistas, $L_\text{p}$; la potencia asociada a electrones relativistas, $L_\text{e}$; y la potencia en el campo magnético, $L_\text{B}$. Luego


\begin{equation}\label{eq:mod-11}
    L_\text{j}=L_\text{p}+L_\text{e}+L_\text{B}.
\end{equation}

Podemos desarrollar las expresiones de estas potencias en función de las variables anteriores. Para la potencia asociada a la masa no relativista $L_p$

\begin{equation}\label{eq:mod-12}
    L_p=\pi R^2\Gamma_\text{j}^2\beta_\text{j}c^3\rho'=\Gamma_\text{j}\dot{M}c^2,
\end{equation}
para la potencia transportada electrones relativistas $L_e$
\begin{equation}\label{eq:mod-13}
    L_e = \pi R^2 \Gamma_j^2 \beta_j c \, (U'_e + p'_e)
\end{equation}
con $U'_e$ la densidad de energía y $p'_e=U'_e/3$ la presi\'on de los electrones relativistas; notamos que podemos escribir la entalpía $h'$ en función de $L_e$ y $L_p$ como $h'=1+L_e/L_p$; finalmente la potencia transportada por el campo magnético $L_B$
\begin{equation}\label{eq:mod-14}
    L_B = \pi R^2 \Gamma_j^2 \beta_j c \, (U'_B + p'_B)
\end{equation}
donde $U'_B = B'^2 / 8\pi$ es la densidad de energía magnética y $p'_B$ la presión magnética; aprovechamos para reescribir la magnetizaci\'on como $\sigma'=L_\text{B}/(L_\text{e}+L_\text{p})$

Dado que en la base la magnetizaci\'on es alta, tenemos que $L_\text{B}$ domina la distribuci\'on de potencia total cerca de la base y disminuye a medida que $z$ aumenta, convirtiendo gradualmente la energ\'ia magnética en aceleración del flujo y energ\'ia de partículas relativistas, y aumentando $L_\text{e}$ y $L_\text{p}$

\begin{figure}[h!]
    \centering
    \begin{subfigure}[b]{0.48\textwidth}
\centering
    \includegraphics[width=\linewidth]{graphics/Modelo/Jet_Power.jpeg}
    \caption[Balance en la potencia a lo largo del \jet]{Figura ilustrativa del balance en las componentes de la potencia a lo largo del \jet}
    \end{subfigure}
    \begin{subfigure}[b]{0.48\textwidth}
\centering
    \includegraphics[width=0.75\linewidth]{graphics/Modelo/sigma_gamma.jpeg}
     \caption[Evolucion de la magnetizaci\'on y el factor de Lorentz a lo largo del \jet]{Figura ilustrativa de la evolucion de la magnetizaci\'on y el factor de Lorentz a lo largo del \jet}
    \end{subfigure}
    \label{fig:modelo-potencia}
\end{figure}

\subsection{Potencia inyectada}
Para caracterizar la inyección en partículas relativistas suponemos que una fracción fija $\varepsilon_\text{e}$ de la potencia total del \jet se transfiere a electrones no térmicos en el rango comprendido entre $z_0$ y $z_\text{max,iny}$. La potencia disponible para inyección resulta entonces  $L_\text{iny} = \varepsilon_\text{e} L_\text{j}$.

Parametrizamos la dependencia espacial de la tasa de inyección mediante
\begin{equation}\label{eq:mod-15}
    \frac{dL_\text{iny}(z)}{dz}=\frac{L_0}{z_0}\left(\frac{z}{z_0}\right)^{-(1+a_\text{iny})}
\end{equation}
donde $a_\text{iny}$ controla la distribución longitudinal de la inyección y $L_0$ es una constante de normalización. La condición de conservación de la potencia inyectada en el intervalo $[z_0, z_\text{max,iny}]$,
\[
    \int_{z_0}^{z_\text{max,iny}} \frac{dL_\text{iny}(z)}{dz}\,dz = L_\text{iny},
\]
permite obtener 
\begin{equation}\label{eq:mod-16}
    L_0=\frac{a_\text{iny}L_\text{iny}}{1-(z_0/z_\text{max,iny})^{a_\text{iny}}},
    \qquad a_\text{iny} \neq 0.
\end{equation}
En el caso particular $a_\text{iny}=0$, la inyección es constante por unidad de $\log(z)$ a lo largo del \jet.  

La potencia total acumulada en electrones no térmicos hasta una distancia $z$ se obtiene integrando la expresión \eqref{eq:mod-16} entre $z_0$ y $z$, lo que conduce a  

\begin{equation}\label{eq:mod-17}
    L_\text{e}(z_0,z)=L_\text{iny}\frac{1-(z_0/z)^{a_\text{iny}}}
    {1-(z_0/z_\text{max,iny})^{a_\text{iny}}}.
\end{equation}

\section{Distribución de partículas relativistas.}

En cada segmento del \jet se inyecta una población de electrones relativistas con una distribución de potencias y un corte exponencial en altas energías. En el sistema comovil,

\begin{equation}\label{eq:mod-18}
    q'(\gamma', z) = q'_0(z)\, \gamma'^{-p} \exp\left[-\frac{\gamma'}{\gamma'_{\max}(z)}\right]
\end{equation}

donde $p$ es el índice espectral (típicamente $2 \lesssim p \lesssim 3$, considerado constante a lo largo del \jet), $\gamma'_{\max}(z)$ es la energía máxima alcanzada por los electrones en la posición $z$, determinada por el balance local entre aceleración y pérdidas, y $q'_0(z)$ es un factor de normalización. Este último se ajusta de manera que la potencia inyectada satisfaga la ecuación de inyección \eqref{eq:mod-16}

\begin{equation}\label{eq:mod-19}
    \frac{dL_{\text{inj}}(z)}{dz} 
= \Gamma_j(z) \pi R^2(z) q'_0(z) 
\int_{\gamma'_{\min}}^{\gamma'_{\max}(z)} 
\gamma'^{-p} e^{-\gamma'/\gamma'_{\max}(z)} d\gamma'.
\end{equation}
Una vez inyectados, los electrones evolucionan en energía y altura a lo largo del \jet. Su distribución estacionaria $n'(\gamma', z)$ obedece la ecuación de transporte relativista,  

\begin{comment}
\begin{equation}\label{eq:mod-20}
    \frac{1}{\pi R^2(z)} 
\frac{\partial}{\partial z} \Big[ \pi R^2(z) \, \Gamma_j(z) \beta_j(z) c \, n'(\gamma', z) \Big]
+ \frac{\partial}{\partial \gamma'} \Big[ \dot{\gamma}'(\gamma',z) \, n'(\gamma',z) \Big]
= q'(\gamma',z),
\end{equation}
\end{comment}
\begin{equation}\label{eq:mod-20}
    \frac{1}{\pi R^2} 
\frac{\partial}{\partial z} \Big[ \pi R^2 \, \Gamma_j \beta_j c \, n'(\gamma') \Big]
+ \frac{\partial}{\partial \gamma'} \Big[ \dot{\gamma}'(\gamma') \, n'(\gamma') \Big]
= q'(\gamma'),
\end{equation}

donde $\dot{\gamma}'$ representa la tasa de variación de energía de las partículas.  

Para simplificar la notación, introducimos $\tilde{N}'(\gamma',z) = \pi R^2(z)\Gamma_j(z)\beta_j(z)c \, n'(\gamma',z)$ como la densidad espectral de partículas transportada por unidad de altura, $\dot{\gamma}'_z = [\Gamma_j(z)\beta_j(z)c]^{-1}\dot{\gamma}'$ como la tasa de pérdidas por unidad de distancia, y $dQ'/dz=\pi R^2(z)q'(\gamma',z)$ como la tasa de inyección de partículas por unidad de altura. Con estas definiciones, la Ec.~\eqref{eq:mod-20} se puede reescribir como

\begin{equation}\label{eq:mod-21}
    \frac{\partial \tilde{N}'(\gamma',z)}{\partial z} 
+ \frac{\partial}{\partial \gamma'} \left[ \dot{\gamma}'_z(\gamma',z) \tilde{N}'(\gamma',z) \right]
= \frac{dQ'(\gamma',z)}{dz}
\end{equation}

La ecuación de transporte no admite en general una solución analítica cerrada. En este trabajo empleamos un esquema numérico de diferencias finitas siguiendo el método de \cite{Park1996}.


\section{Radiación emitida.}

En este trabajo enfocamos el estudio de la emisión en radio presente en \jets relativistas. Dado que en estas longitudes de onda la emisión se encuentra dominada por radiación sincrotrón, nuestro modelo se enfoca en este tipo de proceso.

\subsection{Flujo observado}\label{subsec:flujo}

El flujo observado representa la cantidad de energía que alcanza al observador por unidad de tiempo, área en una dada frecuencia. Matematicamente, el flujo observado se vincula con la intensidad especifica como la integral sobre el \'angulo solido subtendido por la fuente

\begin{equation}\label{eq:mod-22}
F_\nu=\int_\Omega I_\nu \cos(\theta)d\Omega \approx\int_\Omega I_\nu d\Omega.
\end{equation}

Dado que el ángulo proyectado en el cielo es muy pequeño, aproximamos $\cos(\theta)\approx 1$. El angulo solido $\Omega$ puede vincularse con el \'area proyectada en el cielo ocupada por la fuente seg\'un

\begin{equation*}
\Omega=\frac{A}{d^2} \qquad d\Omega=\frac{dA}{d^2},
\end{equation*}
luego

\begin{equation}\label{eq:mod-23}
F_\nu\approx\int_\Omega I_\nu d\Omega.=\int_A I_\nu \frac{dA}{d^2} =\frac{1}{d^2}\int_A I_\nu dxdy,
\end{equation}
donde consideramos $y=z\sin(i)$ la dirección de propagación del \jet proyectada y $x$ perpendicular a esta. En consecuencia, tendremos que

\begin{equation}\label{eq:mod-24}
F_\nu=\frac{\sin(i)}{d^2}\int\int I(x,z)dxdz.
\end{equation}

La intensidad específica $I_\nu$ puede calcularse a partir de la ecuación de transporte radiativo, que describe la variación de la intensidad de la radiación a lo largo de una línea de visión

\begin{equation}\label{eq:mod-25}
    \frac{dI_\nu}{ds}=-\alpha_\nu I_\nu+j_\nu
\end{equation}
donde $s$ es la distancia a lo largo del rayo de propagación, $\alpha_\nu$ es el coeficiente de absorción, que permite calcular la atenuación en la intensidad, y $j_\nu$ es el coeficiente de emisión, que representa la cantidad de intensidad especifica añadida al haz por unidad de longitud. Bajo el supuesto de que tanto $j_\nu$ como $\alpha_\nu$ permanecen aproximadamente constantes a lo largo de la trayectoria, la integración de la Ec.~\eqref{eq:mod-25} conduce a

\begin{equation}\label{eq:mod-26}
    I_\nu(x,z)=\frac{j_\nu}{\alpha_\nu}(1-e^{-\tau_\nu(x,z)})
\end{equation}
con $\tau_\nu(x,z)$ la profundidad óptica.

\subsection{Profundidad óptica}\label{ssec:tau}

\begin{wrapfigure}{l}{0.45\textwidth}
    \includegraphics[width=0.45\textwidth]{graphics/Modelo/cone.png}
    \caption{Diagrama del cruce de un rayo de luz por el espesor del \jet.}
\end{wrapfigure}

En el modelo adoptado, la distribución de partículas se encuentra confinada entre los radios interno y externo definidos en cada altura $z$. Para un rayo que atraviesa un espesor $s(x, z)$ del \jet, la cantidad de materia interceptada depende de la distancia transversal $x$ al eje de simetría.
\begin{wrapfigure}{h}{0.4\textwidth}
    \centering
    \includegraphics[width=0.35\textwidth]{graphics/Modelo/schematic.png}
    \label{fig:schematic}
    \caption{Sección transversal del \jet en el plano del cielo.}
\end{wrapfigure}

Asumimos que, dentro de la región comprendida entre los bordes interno y externo, el coeficiente de absorción $\alpha_\nu$ es constante, y que fuera de esta zona la absorción es despreciable. La longitud efectiva recorrida dentro del medio absorbente depende de la posición transversal $x$, ya que algunos rayos atraviesan solo la envoltura externa, mientras que otros cruzan ambas regiones. En este caso,

\begin{equation}\label{eq:mod-39}
    s(x, z) = \frac{l(x, z)}{\sin i},
\end{equation}

donde $l(x, z)$ es la longitud proyectada del segmento e $i$ es el ángulo entre la dirección de propagación del \jet y la línea de visión del observador:


\begin{equation}
l(x, z) =
\begin{cases}
    2\sqrt{R_{\mathrm{out}}^2(z) - x^2}, 
    & |x| \ge R_\mathrm{in}(z), \\[1ex]
    2\left[\sqrt{R_{\mathrm{out}}^2(z) - x^2}
    - \sqrt{R_{\mathrm{in}}^2(z) - x^2}\right], 
    & |x| < R_\mathrm{in}(z).
\end{cases}
\label{eq:mod-40}
\end{equation}

Con esto podemos resolver la Ec.~\eqref{eq:mod-26} para la {intensidad espectral} $I_\nu$ a lo largo de un trayecto $l(x, z)$, definido por la Ec.~\eqref{eq:mod-40}:

\begin{equation}\label{eq:mod-37}
    I_\nu(x, z) = \frac{j_\nu(z)}{\alpha_\nu(z)}
    \left[ 1 - e^{-\alpha_\nu(z)\, l(x, z)} \right].
\end{equation}
Aqui presentamos $j_\nu$ y $\alpha_\nu$ en el sistema del observador. Podemos transformarlos al sistema comovil utilizando las invariantes relativistas $j_\nu/\nu^2$, $\alpha_\nu~\nu$. Dado que la frecuencia transforma entre sistemas de referencia seg\'un el factor de Doppler (Ec. \eqref{eq:marco-9} y \eqref{eq:marco-10}). Luego el coeficiente de emisi\'on transforma seg\'un

\begin{equation*}
\frac{j'_{\nu'}}{\nu'^2} =\frac{j_\nu}{\nu^2} ,
\end{equation*}
\begin{equation}\label{eq:mod-37b}
\frac{j'_{\nu'}}{\nu'^2} =\frac{j'_{\nu'}}{(\nu/\delta)^2}= \frac{j'_{\nu'} \delta^2}{\nu},
\end{equation}
\begin{equation*}
j'_{\nu'} = j_\nu / \delta^2,
\end{equation*}
mientras que el coeficiente de absorci\'on resulta
\begin{equation*}\label{eq:mod-37c}
\alpha'_{\nu'}~\nu' = \alpha_\nu~\nu,
\end{equation*}
\begin{equation}
\alpha'_{\nu'} = \alpha'_{\nu'}~(\nu/\delta)=\frac{\alpha'_{\nu'}}{\delta}~\nu,
\end{equation}
\begin{equation*}
\alpha'_\nu = \delta\, \alpha_\nu.
\end{equation*}
De estos resultados la expresi\'on de la intensidad en el sistema comovil resulta



\begin{equation}
     I'_\nu'(x, z) = \frac{j'_{\nu'}(z)}{\alpha'_{\nu'}(z)}\, \delta^3
    \left[ 1 - e^{-\frac{\alpha'_{\nu'}(z)}{\delta}\, l(x, z)} \right].
    \label{eq:mod-38}
\end{equation}


\subsection{Coeficientes de emisividad y absorci\'on}
La {emisividad sincrotrón} $j_\nu$ describe la potencia radiada por unidad de volumen, frecuencia y ángulo sólido:

\begin{equation}\label{eq:mod-32}
    j_\nu = \frac{dE}{dt\, d\nu\, dV\, d\Omega}.
\end{equation}

En el sistema comovil, para un valor fijo de $z$ y energía de fotón $E_\gamma=h~\nu'$, la potencia emitida por un único electrón de energía $E'$ en un campo magnético $B'(z)$ puede calcularse como se detalla por la Ec.~\eqref{eq:mod-31} detallada en el Anexo \ref{cha:apx1}. La emisividad total se obtiene multiplicando dicha potencia por la distribución de electrones $N'(E', z)$ e integrando sobre todas las energías disponibles. La importancia de utilizar el sistema comovil reside en que nos permite asumir razonablemente que la emisividad resulta isotropica en este, por lo cual resulta

\begin{equation}
    j'_{\nu'}(z) = \frac{1}{4\pi} 
    \int P(\nu', E', B'(z))\, N'(E', z)\, dE'.
    \label{eq:mod-33}
\end{equation}

La {absorción sincrotrón} se caracteriza mediante el coeficiente $\alpha_\nu$, que cuantifica la atenuación de la radiación al propagarse a través del plasma. Su cálculo en el sistema comovil se basa en la derivada espectral de la distribución de electrones en espacio logarítmico,

\begin{equation}\label{eq:mod-34}
    \frac{d \ln N'}{d \ln E'},
\end{equation}
a partir de la cual, en combinaci\'on con la potencia sincrotron (Ec. \eqref{eq:mod-31}) se construye el integrando

\begin{equation}\label{eq:mod-35}
    P_{\nu'}(\nu', E', B'(z))\, N'(E', z)\, 
    \left( \frac{d \ln N'}{d \ln E'} - 2 \right).
\end{equation}
Asumiendo isotropia (nuevamente, razonable en el sistema comovil) la integración sobre energía, multiplicada por el factor físico $-h^3 c^2 / (8\pi E_\gamma^2)$, conduce a la expresión del coeficiente de absorción en el sistema comóvil:

\begin{equation}
    \alpha'_{\nu'}(z) = 
    -\frac{h^3 c^2}{8 \pi E_\gamma^2}
    \int P_{\nu'}(\nu', E', B'(z))\, N'(E', z)\, 
    \left( \frac{d \ln N'}{d \ln E'} - 2 \right) dE'.
    \label{eq:mod-36}
\end{equation}

\subsection{Regimenes opticos.}

Para calcular el flujo total, integramos la intensidad sobre la coordenada $x$ (Ec.~\ref{eq:mod-24}). En cada segmento del \jet, el tratamiento depende del régimen óptico: delgado, grueso o intermedio (tratamiento general).

\subsubsection*{Tratamiento general}

En el caso general, fuera de los limites de los regímenes ópticamente delgado y grueso, integramos numéricamente para obtener el valor del flujo

\begin{equation*}\label{eq:mod-44}
    \frac{dF_{\nu'}}{dz}
    = \frac{\sin i}{d^2}\,
      \frac{j'_{\nu'}(z)}{\alpha'_{\nu'}(z)}\, \delta^3
      \int_{-R_\mathrm{out}}^{R_\mathrm{out}} (1-e^{-\tau'_{\nu'}(x, z)})\, dx,
\end{equation*}
\begin{equation}
    = \frac{\sin i}{d^2}\,
      \frac{j'_{\nu'}(z)}{\alpha'_{\nu'}(z)}\, \delta^3
      \int_{-R_\mathrm{out}}^{R_\mathrm{out}} \left[ 1 - e^{-\frac{\alpha'_{\nu'}(z)}{\delta \sin(i)}\, l(x, z)} \right] dx,
\end{equation}


\subsubsection*{Caso ópticamente delgado}

La profundidad óptica máxima corresponde al segmento de mayor longitud dentro de la región absorbente, es decir, para $x = R_\mathrm{in}$. En tal caso:

\begin{equation}\label{eq:mod-41}
    \tau_{\nu,\mathrm{max}}(z)
    = \frac{\alpha_\nu(z)}{\sin i}
      \, 2\sqrt{R_{\mathrm{out}}^2(z) - R_{\mathrm{in}}^2(z)}.
\end{equation}

Consideramos que el medio es ópticamente delgado si $\tau_{\nu,\mathrm{max}} < 0.1$. En este régimen, $(1 - e^{-\tau_\nu}) \approx \tau_\nu(x, z)$, y sustituyendo en la Ec.~\ref{eq:mod-24} obtenemos:

\[\]
\begin{equation*}\label{eq:mod-42}
    \frac{dF_{\nu'}}{dz}
    = \frac{\sin i}{d^2}\,
      \frac{j'_{\nu'}(z)}{\alpha'_{\nu'}(z)}\, \delta^3
      \int_{-R_\mathrm{out}}^{R_\mathrm{out}} \tau'_{\nu'}(x, z)\, dx,
\end{equation*}

\begin{equation}
    = \frac{\sin i}{d^2}\, j'_{\nu'}(z)\, \delta^2
      \int_{-R_\mathrm{out}}^{R_\mathrm{out}} l(x, z)\, dx,
\end{equation}

\begin{equation*}
    = \frac{\pi}{d^2}\, j'_{\nu'}(z)\, \delta^2
      \left[R_\mathrm{out}^2(z) - R_\mathrm{in}^2(z)\right].
\end{equation*}

\subsubsection*{Caso ópticamente grueso}

Para un valor medio $x = (R_\mathrm{in} + R_\mathrm{out})/2$, definimos una profundidad óptica promedio $\tau_{\nu,\mathrm{av}}(z) = \alpha'_{\nu'}\, l_\mathrm{av}/(\delta \sin i)$. Si $\tau'_{{\nu'},\mathrm{av}}(z) > 10$, el medio se considera ópticamente grueso y $(1 - e^{-\tau_\nu}) \approx 1$. En este régimen, la contribución al flujo es:

\begin{equation}\label{eq:mod-43}
    \frac{dF_\nu}{dz}
    = \frac{\sin i}{d^2}\,
      \frac{j'_{\nu'}(z)}{\alpha'_{\nu'}(z)}\, \delta^3
      \int_{-R_\mathrm{out}}^{R_\mathrm{out}} dx,
\end{equation}

\begin{equation*}
    = 2R_\mathrm{out}\,
      \frac{\sin i}{d^2}\,
      \frac{j'_{\nu'}(z)}{\alpha'_{\nu'}(z)}\, \delta^3.
\end{equation*}

\section{Convoluci\'on}\label{Sec:Convolucion}

En radioastronomía, un mapa $F(x,y)$ representa una distribución de brillo superficial, en este caso expresada en unidades de flujo por unidad de ángulo sólido. Sin embargo, la resolución finita de los telescopios introduce limitaciones que impiden observar fuentes puntuales de manera perfecta.

La función de dispersión de punto (PSF, por sus siglas en inglés) describe la respuesta de un sistema de imagen ante una fuente puntual u objeto puntual. Esta función se modela comúnmente mediante un haz gaussiano de la forma

\begin{equation}\label{eq:mod-45}
B(x,y) = \exp{\left[-\frac{x^2}{2\sigma_x^2} - \frac{y^2}{2\sigma_y^2}\right]},
\end{equation}
donde $\sigma_x$ y $\sigma_y$ caracterizan la extensión del haz en las direcciones $x$ e $y$, respectivamente.
El mapa observado se obtiene entonces como la convolución del mapa modelado con el haz instrumental.

\subsection{Convolucion discreta.}

Para reproducir las condiciones de observación interferométrica, tomamos el mapa de flujo obtenido (ver Subsec. \ref{subsec:flujo}) y realizamos una convolución discreta sobre las celdas espaciales, dada por

\begin{equation}\label{eq:mod-46}
F_\mathrm{conv}(x',y') \approx \sum_{x,y} F(x,y), B(x-x',y-y'), \Delta x, \Delta y,
\end{equation}
donde $\Delta x$ y $\Delta y$ representan el tamaño de las celdas. Estos valores se definen como $\Delta x, \Delta y = f_\mathrm{Ny}, \sigma_{x,y}$, siendo $f_\mathrm{Ny}$ la frecuencia de muestreo de Nyquist. De esta forma, el flujo observado convolucionado resulta

\begin{equation}\label{eq:mod-47}
F_\mathrm{conv}(x',y') \approx \sum_{x,y} F(x,y),
\exp!\left[-\frac{(x-x')^2}{2\sigma_x^2} - \frac{(y-y')^2}{2\sigma_y^2}\right]
, \Delta x, \Delta y.
\end{equation}

Finalmente, es posible calcular el cociente entre el flujo total modelado y el flujo total observado. Dado el flujo modelado por celda $F(x,y)$, el flujo total se obtiene como

\begin{equation}\label{eq:mod-48}
S = \sum_{x,y} F(x,y), \Delta x, \Delta y.
\end{equation}
Por su parte, el flujo total tras la convolución está dado por

\begin{equation}\label{eq:mod-49}
S_\mathrm{conv} = \sum_{x,y} F(x,y), \frac{\Delta x, \Delta y}{A_\mathrm{haz}},
\end{equation}
donde $A_\mathrm{haz} = 2\pi, \sigma_x, \sigma_y$ corresponde al área del haz gaussiano. El término ${\Delta x, \Delta y}/{A_\mathrm{haz}}$ representa el cociente entre el área de una celda y el área efectiva del haz de observación.

\subsection{Prueba de convolucion sobre dos fuentes puntuales.}

Consideramos dos fuentes puntuales que emiten un flujo de $1~\mathrm{mJy}$ cada una. Se realizó una convolución con un haz gaussiano, suponiendo una anchura a media altura (FWHM, por sus siglas en inglés) de $0.1~\mathrm{mas}$ en el plano del cielo.

\begin{figure}[h]
\centering
\begin{subfigure}[b]{0.48\textwidth}
\centering
\includegraphics[width=\linewidth]{graphics/Modelo/Convolucion_a_dos_puntos.png}
\caption[Mapa de prueba de flujo para dos fuentes puntuales]{Mapa de prueba de flujo para dos fuentes puntuales proyectado en coordenadas $x$ y $z$ en el plano del cielo.}
\label{fig:conv-dos-puntos-S}
\end{subfigure}
\hfill
\begin{subfigure}[b]{0.48\textwidth}
\centering
\includegraphics[width=\linewidth]{graphics/Modelo/Convolucion_a_dos_puntos_Tb.png}
\caption[Mapa de prueba de temperatura de brillo para dos fuentes puntuales]{Mapa de prueba de temperatura de brillo para dos fuentes puntuales proyectado en coordenadas $x$ y $z$ en el plano del cielo.}
\label{fig:conv-dos-puntos-Tb}
\end{subfigure}
\caption[Prueba de convolución sobre dos fuentes puntuales]{Resultados de la convolución sobre dos fuentes puntuales de igual flujo con un haz gaussiano de FWHM $=0.1~\mathrm{mas}$.}
\label{fig:conv-dos-puntos}
\end{figure}

El flujo total previo a la convolución corresponde naturalmente a la suma de los flujos de ambas fuentes, es decir $S = S_1 + S_2 = 2~\mathrm{mJy}$. Tras la convolución, el flujo total obtenido es $S_\mathrm{conv} \approx 1.941~\mathrm{mJy}$, lo que implica un cociente $S_\mathrm{conv}/S \approx 0.9705$. Dicha ligera disminución resulta consistente con el efecto de suavizado introducido por el haz gaussiano, que redistribuye el flujo sobre un área más extensa del plano del cielo, reproduciendo así las limitaciones impuestas por la resolución del instrumento.


%\clearemptydoublepage
\chapter{Resultados}\label{cha:resultados}

En este capítulo presentamos los resultados obtenidos para el ajuste de parámetros del modelo desarrollado, a partir de las observaciones. 
En primer lugar, presentamos los valores ajustados para las condiciones iniciales según los datos observacionales disponibles en la literatura. A continuación, presentamos nuestros resultados para la distribución espectral de energía de ajuste, así como el mapa de temperatura de brillo proyectado en el cielo. Finalmente, presentamos el perfil de temperatura de brillo en función de la altura.
\begin{comment}
En primer lugar, mostramos la distribución espectral de energía correspondiente a la emisión en radio del \jet. 
Posteriormente, a partir de los datos observacionales reportados por Janssen et al. (2021), ajustamos un mapa de calor de la temperatura de brillo proyectada en el cielo, junto con un perfil de temperatura de brillo en función de la altura. 
Finalmente, comparamos estas distribuciones con los resultados del modelo, variando los parámetros libres hasta obtener un ajuste consistente con las observaciones.
\end{comment}

\section{Parámetros y condiciones iniciales.}

Los valores iniciales de los parámetros dinámicos, geométricos y radiativos del modelo fueron determinados combinando restricciones teóricas con observaciones del objeto analizado. Estos parámetros se eligieron de manera que el modelo reproduzca tanto la potencia total observada del \jet como las características del espectro en radio. 

Los parámetros de la fuente son $d=3.8$ Mpc \citep{Harris2010}, $M_\mathrm{BH} = 5.5\cdot10^7~M_\odot$ \citep{Neumayer2010}, $\theta=48\degree$ \citep{Mueller2011}. La emisión predicha por nuestro modelo es radiación sincrotrón, siendo SSC despreciable. Integramos la emisión del \jet hasta $z < z_\mathrm{max}=10^4 R_g$.

\begin{figure}[h]
    \centering
    \includegraphics[width=0.6\linewidth]{graphics/Resultados/RvZ.jpg}
    \caption[Diámetro del radio interior del \jet contra la altura proyectada en el cielo.]{Diámetro del radio interior del \jet contra la altura proyectada en el cielo. Puntos rojos representan observaciones de \cite{2021NatAs...5.1017J}.}
    \label{fig:result-1}
\end{figure}

Para el ajuste de radios, así como el parámetro de apertura que determina la estructura parabólica del \jet, se utilizaron observaciones de \cite{2021NatAs...5.1017J}. Por medio del perfil de colimación obtenido en dicho trabajo, fijamos el parámetro de apertura $w=0.33$ en la Ec. \eqref{eq:mod-3}, tanto para el radio interno como para el externo.

La velocidad del flujo está mediada por el factor de Lorentz en función de la altura, Ec. \eqref{eq:mod-5}

\begin{equation}\label{eq:result-1}
    \Gamma_\mathrm{j}(z) = \Gamma_{\mathrm{j},0} \left(\frac{z}{z_0}\right)^g,
\end{equation}
tomamos el \jet casi no relativista en la base, de modo que $\Gamma_{j,0}\approx1$. Para el valor del factor de potencias tomamos $g=0.13.$

\textcolor{red}{No estoy seguro que mas decir aca.}\

\section{Distribución espectral de energía.}

Para caracterizar la emisión del \jet modelado calculamos el flujo en función de la altura para cada frecuencia, según lo descrito en la Sección \ref{subsec:flujo}. Luego, integrando las Ecuaciones \eqref{eq:mod-42},\eqref{eq:mod-43}, \eqref{eq:mod-44} en la altura del \jet podemos calcular el flujo total en función de la frecuencia.
\begin{figure}[h!]
    \centering
    \includegraphics[width=0.75\linewidth]{graphics/Resultados/dFdz.png}
    \caption{Flujo emitido por celda en funci\'on de la frecuencia.}
    \label{fig:result-2}
\end{figure}
\begin{figure}[h!]
    \centering
    \includegraphics[width=0.75\linewidth]{graphics/Resultados/fluxTot.png}
    \caption{Flujo integrado emitido por el \jet en funci\'on de la frecuencia}
    \label{fig:result-3}
\end{figure}

En las Figuras \ref{fig:result-2} y \ref{fig:result-3} se muestra la distribución espectral de energía (SED) por celda del \jet, así como el flujo total integrado a lo largo de su eje. Ambas figuras fueron obtenidas mediante el modelo de emisión sincrotrón desarrollado en el capítulo anterior. Dicho modelo reproduce adecuadamente la forma general del espectro, caracterizada por una pendiente suave en el rango de radio a óptico y una caída pronunciada hacia frecuencias más altas, en concordancia con un mecanismo de enfriamiento radiativo dominante.

Esta emisión se origina a partir de la distribución de partículas en función de la energía y la altura en el \jet, $N(E,z)$, la cual se calcula a partir de la Ecuación \ref{eq:mod-20}. Para ello, se adopta una función de inyección de partículas según la Ecuación \ref{eq:mod-17}, con un índice espectral $p = 2.3$.

En la Figura \ref{fig:result-2} se observa que el pico de emisión se desplaza hacia frecuencias progresivamente más bajas al aumentar la altura en el \jet. Este comportamiento sugiere una pérdida gradual de energía de las partículas, consistente con un escenario dominado por procesos de enfriamiento sincrotrón y adiabaticos a medida que el plasma se expande y se aleja de la base del \jet.

Podemos ver en la Figura \ref{fig:result-3} que, a frecuencias más altas, el flujo presenta una caída con pendiente de $\sim -1.15$. Este valor de pendiente es consistente con el esperado para una población de partículas con un índice de energía $p \simeq 2.3$, en concordancia con la función de inyección adoptada en el modelo.


\section{Mapa de temperatura de brillo en el cielo.}

Adicionalmente realizamos un mapa de temperatura de brillo en funcion de las coordenadas en el plano del cielo $(x,y)$ para nuestro modelo. En el caso de fuentes resueltas, como Cen A, este tipo de mapas sintéticos constituye una herramienta complementaria para contrastar los resultados del modelo con los mapas de flujo observados y evaluar la validez de los parámetros físicos adoptados.

\begin{figure}[h!]
    \centering
    % --- Row 1 ---
    \begin{subfigure}[b]{0.49\textwidth}
        \centering
        \includegraphics[width=\textwidth]{graphics/Resultados/Mapa_Tb_0-135.jpeg}
        \caption{Mapa de temperatura de brillo en el cielo, $R_\mathrm{in,0}=0$}
        \label{fig:map1}
    \end{subfigure}
    \hfill
    \begin{subfigure}[b]{0.49\textwidth}
        \centering
        \includegraphics[width=\textwidth]{graphics/Resultados/Mapa_Tb_75-135.jpeg}
        \caption{Mapa de temperatura de brillo en el cielo, $R_\mathrm{out,0}=135R_g~R_\mathrm{in,0}=75R_g$}
        \label{fig:map2}
    \end{subfigure}

    \vspace{0.5cm}

    % --- Row 2 ---
    \begin{subfigure}[b]{0.49\textwidth}
        \centering
        \includegraphics[width=\textwidth]{graphics/Resultados/SQRT_T_brightness_(230GHz).png}
        \caption{Mapa de temperatura de brillo en el cielo, $R_\mathrm{out,0}=135R_g~R_\mathrm{in,0}=100R_g$}
        \label{fig:map3}
    \end{subfigure}
    \hfill
    \begin{subfigure}[b]{0.49\textwidth}
        \centering
        \includegraphics[width=\textwidth]{graphics/Resultados/Mapa_Tb_130-135.jpeg}
        \caption{Mapa de temperatura de brillo en el cielo, $R_\mathrm{out,0}=135R_g~R_\mathrm{in,0}=130R_g$}
        \label{fig:map4}
    \end{subfigure}

    \caption{Mapas de temperatura de brillo para diversos valores de $R_\mathrm{out,0}$, $R_\mathrm{in,0}$}
    \label{fig:result-8}
\end{figure}

La Figura~\ref{fig:result-8} presenta los mapas obtenidos para distintos valores de $R_\mathrm{in,0}$ y $R_\mathrm{out,0}$. En particular, la Figura~\ref{fig:map1} muestra el caso de un \jet completamente lleno en su base (sin cavidad interna). La emisión resultante es simétrica en torno al eje, con un máximo de brillo localizado muy cerca de la base y un decaimiento progresivo a lo largo del eje del flujo.

En contraste, las Figuras~\ref{fig:map2}, \ref{fig:map3} y~\ref{fig:map4} corresponden a modelos con distintos cocientes $R_\mathrm{out,0}/R_\mathrm{in,0}$. En la Figura~\ref{fig:map2} comienza a observarse la aparición de una zona central menos brillante, lo que sugiere un \jet parcialmente hueco. A medida que el cociente $R_\mathrm{out,0}/R_\mathrm{in,0}$ disminuye, el brillo máximo tiende a concentrarse en una capa externa, dando lugar a una morfología en la que la emisión se confina a una franja delgada alrededor del eje. En estos casos, la temperatura de brillo máxima disminuye ligeramente, lo que indica una redistribución del flujo radiado. Finalmente, en la Figura~\ref{fig:map4} la emisión central prácticamente desaparece y el perfil longitudinal se extiende más en altura, aunque con una intensidad total menor.


En la Figura \ref{fig:result-5} se presenta un mapa sintético de la emisión del \jet a 230 GHz, en concordancia con las observaciones reportadas por \cite{2021NatAs...5.1017J}. Para obtener dicho mapa, se integró la emisión del \jet proyectada en el plano del cielo y posteriormente se convolucionó con un haz gaussiano (ver Sec. \ref{Sec:Convolucion}), reproduciendo así las condiciones de observación interferométrica. 

\begin{figure}[H]
    \centering
    \includegraphics[width=0.7\linewidth]{graphics/Resultados/SQRT_T_brightness_(230GHz).png}
    \caption[Mapa de temperatura de brillo proyectado en coordenadas $x,~z$ en el plano del cielo]{Mapa de temperatura de brillo proyectado en coordenadas $x,~z$ en el plano del cielo resultante del ajuste a las observaciones de Cen A.}
    \label{fig:result-5}
\end{figure}

Se observa en la figura que el fenómeno de bordes brillantes mencionado en la Sec. \ref{ssec:estructura}. Los bordes del \jet presentan una temperatura de brillo distinta a la de la región interior, con un incremento notable en el límite del borde interno. Es importante considerar el tamaño del haz gaussiano utilizado para el modelado de la observación, ya que este puede modificar fuertemente el mapa resultante (ver Fig. \ref{fig:result-9})

\begin{figure}[H]
    \centering
    \begin{subfigure}[b]{0.49\textwidth}
        \centering
        \includegraphics[width=\textwidth]{graphics/Resultados/Mapa_Tb_convolucionA.jpeg}
    \end{subfigure}

    \vspace{0.5cm}

    % --- Row 2 ---
    \begin{subfigure}[b]{0.49\textwidth}
        \centering
        \includegraphics[width=\textwidth]{graphics/Resultados/Mapa_Tb_convolucionB.jpeg}
    \end{subfigure}
    \hfill
    \begin{subfigure}[b]{0.49\textwidth}
        \centering
        \includegraphics[width=\textwidth]{graphics/Resultados/Mapa_Tb_convolucionC.jpeg}
    \end{subfigure}

    \caption[Mapas de temperatura de brillo para haces gaussianos de convoluci\'on.]{Mapas de temperatura de brillo para haces gaussianos de convoluci\'on. El haz utilizado para la convoluci\'on se encuentra representado por la elipse gris en la esquina superior izquierda de cada mapa.}
    \label{fig:result-9}
\end{figure}

\section{Perfil de temperatura de brillo.}

En la Figura \ref{fig:result-7} podemos ver diversos perfiles de temperatura de brillo máxima en función de la altura del \jet. Dicha figura incluye ajustes para el cociente entre el radio interno y externo (Fig. \ref{fig:sub1}), pruebas en variación de parámetros en z (Fig. \ref{fig:sub2}), ajuste de la potencia total del \jet  (Fig. \ref{fig:sub3}) y pruebas de ajuste en el parámetro $a_\mathrm{inj}$ de la Ecuaci\'on  \eqref{eq:mod-15} (Fig. \ref{fig:sub4}).
\begin{figure}[h!]
    \centering
    % --- Row 1 ---
    \begin{subfigure}[b]{0.49\textwidth}
        \centering
        \includegraphics[width=\textwidth]{graphics/Resultados/T_b_vs_Z_(prueba1).jpeg}
        \caption{Ajuste del perfil de brillo superficial para distintos valores de $R_\mathrm{out,0}/R_\mathrm{in,0}$}
        \label{fig:sub1}
    \end{subfigure}
    \hfill
    \begin{subfigure}[b]{0.49\textwidth}
        \centering
        \includegraphics[width=\textwidth]{graphics/Resultados/T_b_vs_Z_(prueba2).jpeg}
        \caption{Ajuste del perfil de brillo superficial para distintos valores de $z_\mathrm{max}$, $z_\mathrm{max,inj}$ y $z_\mathrm{0}$, en unidades de $R_g$.}
        \label{fig:sub2}
    \end{subfigure}

    \vspace{0.5cm}

    % --- Row 2 ---
    \begin{subfigure}[b]{0.49\textwidth}
        \centering
        \includegraphics[width=\textwidth]{graphics/Resultados/T_b_vs_Z_(prueba3).jpeg}
        \caption{Ajuste del perfil de brillo superficial para distintos valores de $L_\mathrm{j}$}
        \label{fig:sub3}
    \end{subfigure}
    \hfill
    \begin{subfigure}[b]{0.49\textwidth}
        \centering
        \includegraphics[width=\textwidth]{graphics/Resultados/T_b_vs_Z_(prueba4).jpeg}
        \caption{Ajuste del el perfil de brillo superficial para distintos valores de $a_\mathrm{inj}$}
        \label{fig:sub4}
    \end{subfigure}

    \caption{Pruebas de ajuste de parámetros para el perfil de temperatura de brillo.}
    \label{fig:result-7}
\end{figure}

El cociente $R_\mathrm{out,0}/R_\mathrm{in,0}$ determina el grosor de la envoltura del \jet a lo largo de $z$. Dado que adoptamos un mismo parámetro de apertura $w$ para $R_\mathrm{out}$ y $R_\mathrm{in}$ en la Ec.~\eqref{eq:mod-4}, este cociente permanece constante en toda la extensión del \jet. Como se muestra en la Fig.~\ref{fig:sub1}, los modelos con valores pequeños de $R_\mathrm{out,0}/R_\mathrm{in,0}$ producen un \jet más estrecho, con un pico de emisión más compacto e intenso, ya que la radiación se concentra en una región central más densa y caliente. Por el contrario, al aumentar $R_\mathrm{out,0}/R_\mathrm{in,0}$ el \jet se ensancha y la emisión se distribuye sobre un área mayor, lo que disminuye el brillo máximo y suaviza el perfil transversal. El mejor ajuste a las observaciones se obtiene para valores de $R_\mathrm{out,0}/R_\mathrm{in,0}$ cercanos a 1.5, lo que sugiere una estructura del \jet moderadamente ancha en su base.

La Figura~\ref{fig:sub2} muestra el perfil de temperatura de brillo en función de la altura para distintos ajustes de los parámetros que describen la estructura longitudinal del \jet, tales como la altura máxima del \jet, la altura máxima de la región de inyección y la posición correspondiente a la base del \textit{jet}. Se observa que estos parámetros ejercen una influencia significativa sobre la ubicación del máximo en el perfil de temperatura de brillo. En particular, la altura $z$ del máximo coincide aproximadamente con el valor $z_0$ adoptado en el modelo, con una ligera dependencia adicional de $z_\mathrm{max}$. Asimismo, el parámetro $z_\mathrm{max}$ afecta de manera notable la magnitud de la temperatura de brillo: \jets más extendidos tienden a presentar valores de temperatura de brillo más bajos debido a la mayor dispersión de la emisión a lo largo del eje. Por último, $z_\mathrm{max,inj}$ modula la pendiente de ascenso del perfil: para valores más pequeños de este parámetro, la emisión se concentra en regiones cercanas a la base del \jet, dando lugar a un pico más agudo seguido de un descenso más pronunciado.


Finalmente, las Figuras \ref{fig:sub3} y \ref{fig:sub4} muestran el ajuste del perfil para distintos valores de $L_j$ (ver Ec. \eqref{eq:mod-11}) y $a_\mathrm{inj}$, respectivamente. Podemos ver de forma directa que la magnitud del máximo del perfil posee una dependencia directa con el valor de la potencia total del \jet, asi como una ligera dependencia con el valor del parámetro $a_\mathrm{inj}$. A su vez, $a_\mathrm{inj}$ tiene una proporcionalidad inversa con la posicion del máximo de emisión. Esto ocurre debido a que a mayores valores de $a_\mathrm{inj}$ mayor la dependencia inversa de la tasa de inyecci\'on $dL_\mathrm{inj}/dz$ con la altura $z$, por lo cual 

Finalmente, las Figuras~\ref{fig:sub3} y~\ref{fig:sub4} muestran el ajuste del perfil de temperatura de brillo para distintos valores de la potencia total del \jet, $L_\mathrm{j}$ (ver Ec.~\eqref{eq:mod-11}), y del parámetro $a_\mathrm{inj}$, respectivamente. Se observa de manera directa que la magnitud del máximo del perfil posee una fuerte dependencia con el valor de $L_\mathrm{j}$, mientras que muestra una dependencia más débil respecto a $a_\mathrm{inj}$. El parámetro $a_\mathrm{inj}$, a su vez, presenta una relación inversa con la posición del máximo de emisión. Esto posiblemente resulta una consecuencia de que, para valores mayores de $a_\mathrm{inj}$, la tasa de inyección de energía por unidad de altura, $dL_\mathrm{inj}/dz$, decrece más rapidamente con $z$. En consecuencia, la mayor parte de la energía se deposita en regiones más próximas a la base del \jet, concentrando la emisión en zonas internas y desplazando el máximo del perfil hacia alturas menores. Por el contrario, valores más pequeños de $a_\mathrm{inj}$ implican una inyección más extendida a lo largo del \jet, lo que da lugar a un perfil de emisión más amplio y con un pico desplazado hacia alturas mayores.


En la Figura \ref{fig:result-6} se muestra el perfil de temperatura de brillo máxima en función de la altura obtenido para Cen A. Se observa un buen acuerdo con las observaciones hasta aproximadamente $z \sim 200~R_g$, a partir de lo cual la temperatura de brillo observada disminuye rápidamente con la altura. Este comportamiento se condice con lo mostrado en la Figura \ref{fig:marco-2}, que ilustra las observaciones del EHT. En dicha imagen, el \jet parece desviarse a mayores alturas, lo que podría estar asociado a la disminución de la temperatura de brillo en esa región.

\begin{figure}[h!]
    \centering
    \includegraphics[width=0.75\linewidth]{graphics/Resultados/TvZ.jpeg}
    \caption[Perfil de temperatura de brillo $T_\mathrm{b}$ en función de la altura del \jet.]{Perfil de temperatura de brillo $T_\mathrm{b}$ en función de la altura del \jet ajustado a las observaciones de Cen A. En rojo las observaciones del \jet realizadas por \cite{2021NatAs...5.1017J}. En verde las observaciones del contra-\jet.}
    \label{fig:result-6}
\end{figure}


\endinput
 
\clearemptydoublepage
%\chapter{Conclusiones}\label{cha:conclusiones}

En esta tesis buscamos cuantificar... bla bla

El trabajo de tesis requirió el aprendizaje de técnicas radioastronómicas, incluyendo la reducción de observaciones, el cálculo de... bla bla... Tales herramientas serán de gran utilidad en trabajos futuros.

En base a los resultados obtenidos concluimos que... blaaa

En el futuro se espera... (algo que no ocurrirá pero que podría ocurrir) 
%\clearemptydoublepage

%\appendix 
%\chapter{Radiaci\'on sincrotrón}\label{cha:apx1}

Una partícula cargada inmersa en un campo magnético uniforme \textbf{B} emitirá radiación linealmente polarizada denominada radiación sincrotrón. En ambientes astrofísicos, la radiación sincrotrón puede contribuir al flujo de energía
en radio, en el óptico o incluso rayos X blandos. 

En el l\'imite clásico, la potencia por unidad de energía emitida por una partícula de energía $E=\gamma mc^2$ y carga $e$ en un campo magnético $B$, es 
\begin{equation}\label{eq:marco-2}
    P_{\text{syn}}=\frac{\sqrt{3}e^3B}{4\pi mc^2h}\frac{E_\gamma}{E_c}\int_{{E_\gamma}/{E_c}}^\infty K_{5/3}(\xi)d\xi, \\
\end{equation}
donde
\begin{equation}\label{eq:marco-3}
    E_c=\frac{3heB\sin(\alpha)}{4\pi mc}\gamma^2,
\end{equation}
aquí $\alpha$ es el ángulo formado entre la velocidad de la partícula y el campo magnético, $E_\gamma$ es la energía del fotón radiado y $K_{5/3}$ es la función modificada de Bessel \citep{Blumenthal1970}. El espectro de la radiación sincrotrón alcanza un máximo agudo para $E_\gamma\approx0.29E_c$.

\cite{Aharonian2010} presenta una aproximacion para el valor de esta integral. Para ello proponemos $x={E_\gamma}/{E_c}$, luego redefinimos la integral anterior como

\begin{equation}\label{eq:mod-28}
    F(x)=x\int_x^\infty K_{5/3}(\tau)d\tau.
\end{equation}

Si tomamos la componente del campo magnético perpendicular a la velocidad del electrón $B_\perp=B\sin(\theta)$ con $\theta$ el \'angulo entre $\textbf{v}$ y $\textbf{B}$ definimos

\begin{equation}\label{eq:mod-29}
    G(x)=\int\sin(\theta)F\left(\frac{x}{\sin(\theta)}\right)\frac{d\Omega}{4\pi}=\frac{1}{2}\int_0^\pi F\left(\frac{x}{\sin(\theta)}\right)\sin^2(\theta)d\theta.
\end{equation}
Luego proponemos la siguiente aproximación para $G(x)$, con una precisión mejor del $0.2\%$ en todo el rango de $x$.

\begin{equation}\label{eq:mod-30}
    G(x) \approx \frac{1.808x^{1/3}}{\sqrt{1 + 3.4x^{2/3}}}\left(\frac{1 + 2.21x^{2/3}+0.347x^{4/3}}{1 +  1.353x^{2/3}+ 0.217x^{4/3}}\right)e^{-x }.
\end{equation}

Finalmente, la potencia sincrotrón resulta

\begin{equation}\label{eq:mod-31}
    P_{\text{syn}}\simeq\frac{\sqrt{3}e^3B}{4\pi mc^2h}G(x), 
    \qquad x=\frac{E_\gamma}{E_c}.
\end{equation}


La tasa de pérdida de energía para una partícula es

\begin{equation}\label{eq:marco-4}
    \left(\frac{dE}{dt}\right)_{\text{syn}}=-\frac{4}{3}\left(\frac{m_{e}}{m}\right)^2c\sigma_{Th}\frac{B^2}{8\pi}\gamma^2
\end{equation}
donde $m_e$ es la masa en reposo del electrón y $\sigma_{Th}$ es la sección eficaz de Thomson. Una característica clave de la radiación sincrotrón es que se emite naturalmente colimada en un cono de semiapertura $\sim1/\gamma$ respecto a la dirección de movimiento de la partícula.

Para una distribución de partículas $N(E,\alpha)$, la potencia total sincrotrón por unidad de volumen resulta

\begin{equation}\label{eq:marco-5}
    q_{\text{syn}}(E_\gamma)=\int_{E_{min}}^{E_{max}}\int_{\Omega_{\alpha}} N(E,\alpha)P_{syn}(E_\gamma,E,\alpha)\sin(\alpha)dEd\Omega_{\alpha}.
\end{equation}


En particular, si la distribución de electrones es isotrópica y sigue una ley de potencias con índice $p$ ($N(E)\propto E^{-p}$), la integración en energía conduce a un espectro sincrotrón también en forma de potencia:

\begin{equation}\label{eq:marco-6}
    q_{\text{syn}}(E_\gamma)\propto E_\gamma^{\frac{p-1}{2}}.
\end{equation}

Este caso resulta especialmente relevante ya que los principales mecanismos de aceleración de partículas en astrofísica producen de manera natural distribuciones de electrones con índices espectrales cercanos a $p\simeq 2$. En consecuencia, la forma de ley de potencias en la radiación sincrotrón constituye una huella directa de la aceleración no térmica en \jets relativistas y otras fuentes compactas.

\section{Auto-absorci\'on sincrotrón (SSA)}

La emisión sincrotrón suele estar acompañada por procesos de absorción y de emisión inducida. En el primero, los fotones producidos por las partículas relativistas transfieren su energía a otras cargas en presencia del campo magnético, reduciendo la intensidad observable. En el segundo, la emisión estimulada tiende a amplificar la radiación en direcciones y frecuencias donde esta ya es predominante, modificando la distribución angular y espectral de la emisión total.

Para una población de partículas con una distribución de energías en forma de ley de potencias, $N(E) = K E^{-p}$, el coeficiente de absorción puede escribirse como \citep{{Rybicki1979}:

\begin{equation}\label{eq:marco-7}
\alpha_\nu = \frac{(p+2)c^2}{8\pi\nu^2} \int dE, P(\nu,E)\frac{N(E)}{E} \propto \nu^{-(p+4)/2},
\end{equation}
donde $P(\nu, E)$ es la potencia emitida por una partícula de energía $E$ a la frecuencia $\nu$.

Si consideramos una región uniforme en la cual tanto el coeficiente de emisión $j_\nu$ como el de absorción $\alpha_\nu$ son aproximadamente constantes a lo largo de una trayectoria efectiva de longitud $l$, la intensidad emergente puede expresarse como

\begin{equation}\label{eq:marco-8}
I_\nu = \frac{j_\nu}{\alpha_\nu}\left(1 - e^{-\alpha_\nu l}\right),
\end{equation}
En el régimen ópticamente grueso ($\alpha_\nu l \gg 1$) resulta en un espectro con una pendiente característica $I_\nu \propto \nu^{5/2}$, independiente del índice espectral $p$.
Este comportamiento genera un quiebre en el espectro a bajas frecuencias, que marca la transición entre las regiones ópticamente gruesa y delgada del flujo sincrotrón. 

\begin{figure}[h!]
\centering
\includegraphics[width=0.65\linewidth]{graphics/Marco/Sincrotron_absorcion.png}
\caption[Espectro típico de emisión sincrotrón]{Espectro esquemático de emisión sincrotrón mostrando las regiones ópticamente gruesa ($I_\nu \propto \nu^{5/2}$) y ópticamente delgada ($I_\nu \propto \nu^{-(p-1)/2}$). Imagen obtenida de \cite{Rybicki1979}}
\label{fig:synchrotron_break}
\end{figure}



\endinput

\clearemptydoublepage
%\chapter{Programas desarrollados}\label{cha:apx3}

Los programas utilizados para la obtención de los resultados presentados en este trabajo fueron elaborados en Python y se encuentran disponibles en \url{https://github.com/estudiante/proyecto_de_tesis}. A continuación, se explica el funcionamiento de estos programas y cómo emplearlos para replicar los resultados aquí presentados o para su aplicación a otros conjuntos de datos. 


\endinput
%\clearemptydoublepage
%\chapter{Resumen de trabajos realizados}\label{cha:publicaciones}

%----------------------------------------------------------------------------------
%
\section{Publicaciones}
%
%----------------------------------------------------------------------------------

\subsection{Publicaciones con referato en revistas internacionales}

  \begin{enumerate}
  
    \item {\textbf{Título paper1}
    \\autor1, autor2...
    \\revista, 20XX}
    
    \item {\textbf{Título paper2}
    \\autor1, autor2...
    \\revista, 20XX}
    
  \end{enumerate}
  
\subsection{Actas en congresos internacionales}
 
  \begin{enumerate}
    
    \item {\textbf{Título acta1}
    \\autor1, autor2, ...
    \\Proceeding/boletín, 20XX}
    
 \end{enumerate}
  
\subsection{Actas en congresos nacionales}

  \begin{enumerate}
  
    \item {\textbf{Título acta1}
    \\autor1, autor2, ...
    \\Bol. Asoc. Arg. Astron., XX, YY, 20ZZ.}

  \end{enumerate}
  
%----------------------------------------------------------------------------------
%
\section{Propuestas observacionales aprobadas}
%
%----------------------------------------------------------------------------------

\begin{enumerate}
 
    \item Co-I de Propuesta a ... \textit{Título propuesta 1} (20XX). Tiempo asignado: YY hs.
    
    \item PI de Propuesta a ... \textit{Título propuesta 2} (20XX). Tiempo asignado: YY hs.

\end{enumerate}


\endinput


\backmatter
\addcontentsline{toc}{chapter}{\bibname}
\bibliography{biblio/biblio.bib} %pueden ser varios archivos si se gusta: {biblio/biblio1, biblio/biblio2, biblio/biblio3}
%\bibliographystyle{biblio/abbrv.bst}
%\bibliographystyle{biblio/mnras}
%\bibliographystyle{myIEEEtran}

%
%\bibliographystyle{aa} % style aa.bst
\bibliography{biblio} % your references Yourfile.bib

\begin{thebibliography}{}
%
%  \bibitem{heath} Heath, T. 1921, 
%    The Theorem of Pythagoras. A History of Greek Mathematics (2 vols.),
%     Clarendon Press, Oxford.
%
%  \bibitem{maxwell} Maxwell, J. 1865, 
%      Philosophical Transactions of the Royal Society of London, 155, 459--512.
% 
%  \bibitem{newton} Newton, I. 1726, 
%      Philosophiae naturalis principia mathematica, 3"er edición.
%  
%  \bibitem{sandifer} Sandifer, E. 2006, 
%       The Early Mathematics of Leonhard Euler, MAA.
%
   \bibitem{Blasi2013} Blasi, P. 2013, Physical Review, 28, 6, 1049--1070.

\end{thebibliography}

\endinput


%\addcontentsline{toc}{chapter}{\bibname}
%\bibliography{biblio,biblio_BS,biblio_CWB,biblio_MQs,biblio_pol,biblio_JCI}


\end{document}
